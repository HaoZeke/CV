\documentclass[11pt,a4paper, final, factor=1100, stretch=18, shrink=18]{moderncv}

\usepackage{color}
\usepackage{fontspec}

% Stop bugging me about \es
\usepackage{xspace}
% moderncv themes
\moderncvstyle{classic}                             % style options are 'casual' (default), 'classic', 'banking', 'oldstyle' and 'fancy'
%\moderncvcolor{blue}                               % color options 'black', 'blue' (default), 'burgundy', 'green', 'grey', 'orange', 'purple' and 'red'

% the (custom) color which will be used in the cv
\definecolor{color1}{RGB}{1, 52, 64}

% scale the page layout (depreciated for more complicated stuff)
%\usepackage[scale=0.75]{geometry}

% change width of the column with the dates
\setlength{\hintscolumnwidth}{2.5cm}

%\PassOptionsToPackage[final, factor=1100, stretch=18, shrink=18 ]{microtype}
% The final option overrides global defaults. It greatly improves general appearance of the text. The stretch and shrink reduce bluriness[20,20 default]. The factor increases protrusion amount by 10%, [default 1000]
% Tracking allows for small caps, like in cites to be adjusted. The activate commands are to set protrusion.

\microtypecontext{spacing=nonfrench} 						% To preserve interword spacing via \nonfrenchspacing.

\SetExtraKerning[unit=space] 								% These produce more effects, from microtype with kerning.
{encoding={*}, family={bch}, series={*}, size={footnotesize,small,normalsize}}
{\textendash={400,400}, 								% en-dash, add more space around it
"28={ ,150}, 											% Left bracket, add space from right
"29={150, },											% Right bracket, add space from left
\textquotedblleft={ ,150}, 							% Left quotation mark, space from right
\textquotedblright={150, }} 							% Right quotation mark, space from left


\SetExtraKerning[unit=space]
{encoding={*}, family={qhv}, series={b}, size={large,Large}}
{1={-200,-200},
	\textendash={400,400}}

\SetTracking{encoding={*}, shape=sc}{40} 					% This is for better small caps with tracking and microtype.

\SetProtrusion{encoding={*},family={bch},series={*},size={6,7}}  % This enables better optical views.
{1={ ,750},2={ ,500},3={ ,500},4={ ,500},5={ ,500},
	6={ ,500},7={ ,600},8={ ,500},9={ ,500},0={ ,500}}

% Page settings
\usepackage{geometry}
\geometry{
	a4paper,
	%total={210mm,297mm},
	%l,r,t used to be 15mm
	left=15mm,
	right=15mm,
	top=15mm,
	bottom=10mm,
}

% required when changing page layout lengths
\AtBeginDocument{\recomputelengths}
\usepackage{xunicode}
\usepackage{xltxtra}
% \usepackage[utf8]{inputenc} (Xelatex expects UTF8 anyway)

% I like pretty logos
\usepackage{metalogo}

% Widen the letter spacing for logos
\setlogokern{La}{0.2pt}
\setlogokern{Xe}{0.2pt}
\setlogokern{eL}{0.2pt}
\setlogokern{Te}{0.2pt}
\setlogokern{aT}{0.2pt}
\setlogokern{eX}{0.2pt}

% Better Quotes (depreciated for this usage)
%\usepackage{epigraph}

% Poaching from Espresso's ug.tex

%%%%%%%%%%%%%%%%%%%%%%%%%%%%%%%%%%%%%%%%%%%%%%%%%%
%%%%%%%%%%%%%%%%%%%%%%%%%%%%%%%%%%%%%%%%%%%%%%%%%%
%%%%%%%%% New Commands and Environments %%%%%%%%%%
%%%%%%%%%%%%%%%%%%%%%%%%%%%%%%%%%%%%%%%%%%%%%%%%%%
%%%%%%%%%%%%%%%%%%%%%%%%%%%%%%%%%%%%%%%%%%%%%%%%%%
\newcommand{\es}{\mbox{\textsf{ESPResSo}}\xspace}

% german word break/hyphenation rules that's with ngerman (we use english)
\usepackage[british,UKenglish,USenglish,american]{babel}

% insert dummy text (used in the letter)
%\usepackage{lipsum}

% used for \begin{comment}...\end{comment}
\usepackage{verbatim}

% use guilllemets in bibliography (not german.) Also autostyle superceeded babel
\usepackage[autostyle=true]{csquotes}

\usepackage[%
	sorting=ydnt, %
	minbibnames=8,%
	maxbibnames=99,%
	block=space,%
	bibencoding=utf8,%
	backend=biber,%
	url=false,%
	isbn=false,%
]{biblatex}

\bibliography{publications}

% Vanity (https://tex.stackexchange.com/questions/359311/how-to-underline-name-of-specific-authors-in-biblatex)
\usepackage{xpatch}
\usepackage[normalem]{ulem} %for \uline

\newbox\savenamebox

% Actually I don't use the bold version anymore
\newbibmacro*{name:bbold}[2]{%
  \def\do##1{\iffieldequalstr{hash}{##1}{\setbox\savenamebox\hbox\bgroup \listbreak}{}}%
  % \def\do##1{\iffieldequalstr{hash}{##1}{\bfseries\setbox\savenamebox\hbox\bgroup \listbreak}{}}%
  \dolistloop{\boldnames}%
}

\newbibmacro*{name:ebold}[2]{%
  \def\do##1{\iffieldequalstr{hash}{##1}{\egroup\uline{\usebox\savenamebox}\listbreak}{}}%
  \dolistloop{\boldnames}%
}

\xpatchbibmacro{name:given-family}{\usebibmacro{name:delim}{#2#3#1}}{\usebibmacro{name:delim}{#2#3#1}\begingroup\usebibmacro{name:bbold}{#1}{#2}}{}{}

%\xpretobibmacro{name:given-family}{\begingroup\usebibmacro{name:bbold}{#1}{#2}}{}{}
\xapptobibmacro{name:delim}{\begingroup\normalfont}{}{}

\xapptobibmacro{name:given-family}{\usebibmacro{name:ebold}{#1}{#2}\endgroup}{}{}
\xapptobibmacro{name:delim}{\endgroup}{}{}

% Hash in the bbl file
\newcommand*{\boldnames}{}
\forcsvlist{\listadd\boldnames}{
  {0777a94fb5d43a9efe6c5aa525bcd0a8}, % InBold
  {f695e364a633e61d21101f4f462d9373}, % Also me
  }



% get rid of the number-labels ([1], [2], etc.) in front of publications
\defbibenvironment{midbib}
{\list
	{}
	{
		\setlength{\leftmargin}{0mm}
		\setlength{\itemindent}{-\leftmargin}
		\setlength{\itemsep}{\bibitemsep}
		\setlength{\parsep}{\bibparsep}}
}
{\endlist}
{\item}


% add [DOI] and [PDF] fields at the end of each publication entry
\DeclareSourcemap{
	\maps[datatype=bibtex]{
		% the bibtex entry 'mydoi' gets mapped to 'usera'
		\map{
			\step[fieldsource=mydoi]
			\step[fieldset=usera, origfieldval]
		}
		% the bibtex entry 'mypdf' gets mapped to 'usera'
		\map{
			\step[fieldsource=mypdf]
			\step[fieldset=userb, origfieldval]
		}
	}
}

% [DOI] entries in publication
\DeclareFieldFormat{usera}{\color{color1}[\href{#1}{\textsc{doi}}]}
\AtEveryBibitem{
	% put [DOI] stuff at the end of a publication entry
	\csappto{blx@bbx@\thefield{entrytype}}{%
		\iffieldundef{usera}{
			% this gets invoked, once nothing is supplied
			% via the mypdf or mydoi value.
			% you could e.g. display a default thing here.
		}{\space\printfield{usera}}}
}

% [PDF] entries in publication
\DeclareFieldFormat{userb}{\color{color1}[\href{#1}{\textsc{pdf}}]}

\AtEveryBibitem{
	% put [DOI] stuff at the end of a publication entry
	\csappto{blx@bbx@\thefield{entrytype}}{\iffieldundef{userb}{}{\printfield{userb}}}
}

\renewcommand*{\mkbibnamegiven}[1]{%
	\ifitemannotation{highlight}
	{\textbf{#1}}
	{#1}}

\renewcommand*{\mkbibnamefamily}[1]{%
	\ifitemannotation{highlight}
	{\textbf{#1}}
	{#1}
}

% Minion Pro is used as the main font, if you don't
% have it installed uncomment this line or choose another pretty font.
% Literata is also included.
%\setmainfont[Numbers=OldStyle]{Minion Pro}

\setmainfont[
	Path = Fonts/MinionPro/,
	Numbers=OldStyle,
	Extension = .otf,
	UprightFont = *_Regular,
	ItalicFont = *_It,
	BoldFont = *_Bold,
	BoldItalicFont = *_BoldIt
]{MinionPro}

% Myriad Pro is used as the sans font, if you don't
% have it installed uncomment this line or choose another pretty font.
\setsansfont[
	Path = Fonts/MyriadPro/,
	Numbers=OldStyle,
	Extension = .otf,
	UprightFont = *_Regular,
	ItalicFont = *_It,
	BoldFont = *_Bold,
	BoldItalicFont = *_BoldIt,
	Ligatures=TeX,
	Scale=MatchLowercase]{MyriadPro}

% the moderncv package will populate a lot of the pdf meta-information.
% this can be used to change some of these infos.
\AfterPreamble{\hypersetup{
		pdfcreator={XeLaTeX},
		pdftitle={Rohit Goswami's CV}
	}}

% for the icons (telephone, globe). I found the icons provided by the
% fontawesome package prettier than the standard moderncv icons.
\defaultfontfeatures{
	Path = Fonts/FontAwesome/ }
\usepackage{fontawesome}

% personal data
\firstname{Rohit}
\familyname{Goswami}
\quote{``An unproblematic state is a state without creative thought. It's other name is Death.''\\-- David Deutsch}

% \faEnvelope \faPhone \faGithub \faGlobe

% \address{House No. 646, 35th Lane}{IIT Kanpur, 208016}

% I use the extrainfo command for additional information, since I
% want to use custom icons and have finer control over spacing.
\extrainfo{
% \faPhone\hspace{0.3em}\href{tel:919935135006}{\ttfamily +91 9935135006}\\
{\small\faEnvelope}\hspace{0.3em}\href{mailto:rgoswami@ieee.org}{\ttfamily rgoswami@ieee.org}\\
{\small\faGithub}\hspace{0.3em}\href{https://github.com/haozeke}{\ttfamily haozeke}\\
{\small\faTwitter}\hspace{0.3em}\href{https://twitter.com/rg0swami}{\ttfamily rg0swami}\\
\faGlobe\hspace{0.3em}\href{https://rgoswami.me}{\ttfamily rgoswami.me}
}


% picture, resized to a height of 84pt
\photo[84pt]{Picture/rohit}

% spacing before (sub)sections
\newcommand{\spacesection}{\vspace{0.4cm}}
\newcommand{\spacesubsection}{\vspace{0.2cm}}


%===========================
\begin{document}

% Adapted from https://tex.stackexchange.com/a/170219/130845
\begin{tikzpicture}[remember picture,overlay]
      \node[anchor=north, yshift=-0.25cm] at (current page.north) {\underline{Last
		  updated on \today}};
\end{tikzpicture}

\maketitle

\section{Personal Data}
\cvitem{Name}{Rohit Goswami}
\cvitem{Date Of Birth}{10.08.1995}
\cvitem{Birthplace}{Brookhaven, New York, United States of America}

\spacesection{}
\section{Work Experience}

\cventry{\textsc{2021--present}}{Science Institute}{University of Iceland}{}{Rannís Research Fund Doctoral Grant Awardee}{Principal investigator for the project on ``Magnetic interactions of itinerant electrons modeled using Bayesian machine learning'' supervised by Prof. Hannes Jónsson with Prof. Birgir Hrafnkelsson as a co-supervisor.}
\cventry{\textsc{2021--present}}{Quansight Labs}{}{}{Software Engineer}{Working on foundational FOSS scientific codebases and tools like \texttt{numpy} and \texttt{f2py} to provide holistic maintenance.}
\cventry{\textsc{2019--2020}}{Faculty of Physical Sciences}{University of Iceland}{}{Doctoral Researcher}{Worked with Prof. Hannes Jónsson on Bayesian analysis and machine learning for \emph{ab-inito} quantum chemistry; partially employed on the EU funded ``ReaxPro'' project for potential energy surface saddle search acceleration.}
\cventry{\textsc{2019--2020}}{Department of Chemistry}{Indian Institute of Technology, Kanpur}{}{Senior Project Associate}{I was affiliated to the Femtolab under the project ``Femtosecond Laser Approaches to Quantum Information and Quantum Computation''.}
\cventry{\textsc{2018--2019}}{Department of Chemical Engineering}{Indian Institute Of Technology, Kanpur}{}{Project Associate}{I was associated with the Computational Nanoscience group. Over the course of two centrally funded projects, ``Nucleation On Nanostructured Surfaces Computer Simulation      Studies (SPO/DST/CHE/2017294)'' and ``Advanced Computation Research and Education (SPO/MHRD/CC/20130176)'':
	\begin{itemize}
		\item I worked on the
		      implementation of an enhanced version of the CHILL (CHILL+) algorithm for tracking
		      ice types
		\item Designed a linear discriminant analysis technique for near-surface ice structure determination which is undergoing rigorous testing
		\item Implemented a graph based network connectivity model for ice structures
		\item Spearheaded the development of High Performance GPU accelerated molecular dynamics simulation analysis tools
		\item Worked on the determination of optimal GPU cluster configurations
		\item Designed and administered academic outreach websites
	\end{itemize}  }

\spacesection{}
\section{Education}

%\cventry{\textsc{1948}}{Ph.D. Chemistry}{University Extension}{New York}{\emph{United States}}{}
%\cventry{\textsc{1939--1941}}{M.Sc. Chemistry}{University Extension}{New York}{\emph{United States}}{}
\cventry{\textsc{2019--present}}{Graduate studies}{University of Iceland}{Reykjavík}{\emph{Iceland}}{Current GPA: $9.25$}
\cventry{\textsc{2014--2018}}{B.Tech. Chemical Engineering}{Harcourt Butler Technical University}{Kanpur}{\emph{India}}{First Division (\textsc{Project: } Gas Sweetening Plant Design)}
\cventry{\textsc{2011--2013}}{Intermediate (AISSCE)}{Delhi Public School Kalyanpur}{Kanpur}{\emph{India}}{$87.2\%$ Central Board of Secondary Education (CBSE)}
\cventry{\textsc{2009--2011}}{High School (AISSE)}{Delhi Public School Kalyanpur}{Kanpur}{\emph{India}}{$9.8$ Cumulative Grade Point Average (CGPA) in Central Board of Secondary Education (CBSE)}

\spacesection{}
\section{Voluntary Positions}

\cventry{2022-2023}{NumPy}{Python Software Foundation}{}{GSoC Mentor}{I will be mentoring a student for working on the F2PY frontend}

\cventry{2022-2023}{LFortran}{NumFOCUS}{}{GSoC Mentor}{I intend to (co-)mentor students working on the Abstract Semantic Representation of LFortran}

% TODO: Add details
\cventry{\textsc{2021--2022}}{Early Career Engineers Committee}{Professional Engineering Committee}{}{IOP Representative}{}

% TODO: Add details
\cventry{\textsc{2021--2025}}{Early Career Member Group Committee}{Institute of Physics}{}{Ordinary Member}{}

\cventry{\textsc{2021--2022}}{Forum on Graduate Student Affairs}{American Physics Society}{}{Nominating Committee Member}{As part of the nomination committee, along with the past chair and other members, am responsible for validating and selecting nominees for the ballots of FGSA's elections}

\cventry{\textsc{2021--2022}}{Stúdentablaðsins}{Stúdentaráð Háskóla Íslands}{}{Editorial Team Member}{As a member of the editorial board; it is my pleasurable responsibility to provide guidance to the journalists and support my fellow editors in upholding the standards of the student paper of the University; published since 1943}

\cventry{\textsc{August-October 21}}{Summer of Nix}{NixOS, NLNet, and Tweag}{}{Mentor}{Will provide guidance to a group of five students who are to work towards packaging NLNet packages}

\cventry{\textsc{June-August 21}}{LFortran: Computational Chemistry}{Google Summer of Code}{}{Student Developer}{Worked on implementing compile time intrinsic functions and also other front and middle end tasks towards a minimum viable product which compiles production computational chemistry codebases}

\cventry{\textsc{2021--2022}}{Young Professionals Committee}{American Institute of Chemical Engineers}{}{Publications Subcommittee Chair}{Am responsible for overseeing the preparation of all drafts of YPC publications as well as appointing subcommittee positions in coordination with the Chair.}

\cventry{\textsc{2020--2021}}{Instructor Development Committee}{The Carpentries}{}{Special Projects Chair}{Was tasked with facilitating the formation of task forces/committees as appropriate to accomplish desired project outcomes; in particular with a focus on improving recognition for community contributions}

\cventry{\textsc{Sep 20-March 21}}{Symengine}{Google Season of Docs}{}{Technical Writer}{Was tasked with a long project extending the existing documentation and setting up tests and websites for the Symengine project}

% \cventry{\textsc{September 2020}}{Software Citation Project}{eLife Innovation Sprint 2020}{}{Member}{2 day long development sprint focused on FOSS science and improving the tooling of software as science projects with a focus on the Citation File Format (CFF) and the R2T2 (Research References Tracking Tool) tool}

\cventry{\textsc{2020--present}}{TeX Users Group Conference (TUG202X)}{TeX Users Group}{}{Organizing Committee Member}{Have administered the TUG Zulip and coordinated social activities (including Topia) from the first online TUG, the 41st in 2020 and continuing to support the committee in subsequent interactions}

\cventry{\textsc{2020--2021}}{Executive Student Council}{American Institute of Chemical Engineers}{}{Publications Webmaster}{Was tasked with managing the publications committee web resources.}

\cventry{\textsc{2019--2020}}{IEEE P1940}{IEEE Standards Committee}{}{Working group member}{Was actively engaged in working with stake holders in industry and academia to create a collection of standard profiles that define integration of authentication services with ISO 8583 used for financial transactions.}

\cventry{\textsc{2019--present}}{R Novice Inflammation}{The Carpentries}{}{Maintainer}{As a maintainer for the Software Carpentries lesson on R, I work with the community to make sure that lessons stay up-to-date, accurate, functional and cohesive. }

\cventry{\textsc{2019--2020}}{CarpentryCon 2020}{The Carpentries}{}{Program Committee co-chair \& Website subcommittee member}{Working for an international conference with diverse leads from across the world, as part of the program committee I reached out to keynote speakers and managed the overall schedule. Wrote content with the website subcommittee and also contributed due to my web development expertise.}

\cventry{\textsc{2019--present}}{Univ.ai}{Earth2Orbit Analytix Private Limited}{}{Teaching Fellow and Developer}{Tested course-content and developed interactive labs to work with an online cohort of students. Am presently teaching labs and mentoring small batches. I also work with the front and backend teams to facilitate workflows including shopify stores and NodeJS authentication systems.}

\cventry{\textsc{2018--2019}}{Animal Welfare Group}{Indian Institute of Technology Kanpur}{}{Member and Web-developer}{Have worked with student bodies to rescue and care for local animals. Also designed and maintained a site with ReactJS to enhance knowledge dissemination.}

\spacesection{}
\section{Teaching Experience}
% TODO: Add more carpentries
\cventry{\textsc{May 2022}}{Statistical Inference for Biology}{The Jackson Laboratory}{}{Instructor}{\texttt{R} workshop focused on frequentist statistical inference used for experimental analysis and design.}
\cventry{\textsc{June 2021}}{Web Development for Physicists}{IOP Conference for Astronomy and Physics Students}{}{Invited Instructor}{Overview of web design, development and implementation focusing on technologies useful to the working physicist like SSGs and referencing frameworks}
\cventry{\textsc{April-May 2021}}{CS106A - Code in Place}{Stanford University}{}{Teaching Mentor and Section Leader (TA)}{As
  a returning section leader for the
  \href{https://compedu.stanford.edu/codeinplace/announcement/}{\ttfamily code-in-place
	initiative}, I was given the additional honor of being a mentor to the first time section leaders for this unique course. Also delivered a Nix workshop for the SLs.}
\cventry{\textsc{April 2021}}{C++ part 2 - libraries and simulations}{IOP Student Community}{}{Invited Instructor}{Intensive \texttt{C++} workshop which covered augmenting existing code with \texttt{Python} bindings and using a build automation tool (\texttt{CMake}) along with continuous integration.}
\cventry{\textsc{November 2020}}{Data Carpentry (Social Sciences with R)}{Carnegie Mellon University}{}{Lead Instructor}{\texttt{R} workshop focused on tidy data.}
\cventry{\textsc{September-October 2020}}{Sciware: Git and GitHub}{Flatiron Institute}{}{Lead Instructor}{10 hour long workshop which taught the fundamentals of version control and collaboration using \texttt{git}, with some introductory coverage of more advanced topics like rebases.}
\cventry{\textsc{September 2020}}{Data Carpentry Workshop for Social Sciences}{Georgia Gwinnett College}{}{Lead Instructor}{6 hours on \texttt{R} practices, with a focus on how base-R gives way to tidy forms of data and the \texttt{tidyverse}.}
\cventry{\textsc{July-August 2020}}{Water, Chemicals and more with Computers for Chemistry (WC3m)}{Wave Learning Festival}{}{Co-Teacher}{15 hour long summer course for high school students and undergrads on the basics of computational chemistry.}
\cventry{\textsc{June-July 2020}}{Online Data Carpentry Workshop}{SADiLaR, South Africa}{}{Lead Instructor}{Taught the basics of R, OpenRefine, and some data wrangling to graduate students in the social sciences over three days.}
\cventry{\textsc{June-July 2020}}{Data Carpentry Ecology Workshop}{Biotech Partners}{}{Leading Instructor}{Taught the basics of Python and assisted with shell lessons for high school students over three days, with a follow up mentoring program.}
\cventry{\textsc{May 2020}}{Helper}{CodeRefinery Mega Workshop}{}{}{}
\cventry{\textsc{April-May 2020}}{CS106A - Code in Place}{Stanford University}{}{Section Leader (TA)}{As
  part of the special COVID-19
  \href{https://compedu.stanford.edu/codeinplace/announcement/}{\ttfamily code-in-place
	initiative}, I worked as a section leader (teaching assistant). The course
  covered the fundamentals of computer programming using Python and was built off the first half of CS106A. Also moderated and participated in an AMA session on ``Machine Learning for the Physical Sciences'' and taught a workshop for the other section leaders entitled ``Functional Python Packaging with Nix''}


\spacesection{}
\section{Undergraduate Experience}

\spacesubsection{}
\subsection{Internships}

\cventry{\textsc{2017--2018}}{Dr. Debojit Chakrabarty}{Keva Fragrances Ltd, Mumbai}{}{R\&D Industrial Intern}{Modeling complex multi-component perfumes in a predictive method via experimental and theoretical considerations. In collaboration with Prof. Rajdip Bandyopadhyaya of the ChemE Dept. at IIT Bombay.
}
\cventry{\textsc{Summer 2017}}{Prof. Sibasish Ghosh}{The Institute of Mathematical Sciences, Chennai}{}{Visiting Scholar}{Discussed computational techniques for the simulation and understanding of quantum tomography.}
\cventry{\textsc{Summer 2017}}{Prof. Nisanth Nair}{Indian Institute Of Technology Kanpur}{}{SURGE Scholar}{An exploratory project to understand and deal with bottlenecks in computational chemistry, the major objectives were to investigate hybridization of existing code via OpenMP and MPI.
	\\~\\
	\textsc{Poster: }Development of Computational Tools for Free Energy Calculations of Chemical Reactions}
\cventry{\textsc{Summer 2016}}{Prof. Rajarshi Chakrabarti}{Indian Institute Of Technology Bombay}{}{Research Intern}{Retooled a server with ArchLinux and also simulated patchy colloids (Janus Particles). \\~\\
	\textsc{Project Report: }Computational Survey of Coarse Grained Soft Matter Molecular Dynamics Simulations
}

\spacesubsection{}
\subsection{Volunteer Work}

\cventry{\textsc{2017--2018}}{ChemE Herald}{Harcourt Butler Technical University, Kanpur}{}{Editor-in-Chief}{Inaugurated and managed an interdisciplinary technical newsletter.}

\cventry{\textsc{2017--2018}}{HBTU-MUN 2018}{}{}{Secretary General}{Designed a ReactJS based static website, with Trello backed user registration, also performed outreach pre-events to raise awareness and participation, in addition to overseeing the working of the executive board.}

\cventry{\textsc{2016--2017}}{HBTU-MUN 2017}{}{}{Executive Board Chairperson}{Designed a Jekyll based static website and ensured adherence to standard MUN rules as Chairperson.}

% Sadly this was hit by anti-aircraft guns
% \cventry{\textsc{2016--2017}}{Interface 2017}{Harcourt Butler Technical University, Kanpur}{}{Technical Cell Head}{Organizing and inspiring students to work towards the success of the Department's techno-cultural fest and seminars.}

\cventry{\textsc{2014--2016}}{The Curiosity Magazine}{Harcourt Butler Technical University, Kanpur}{}{Editor-in-Chief}{Managed a diverse team of student content writers and also later typeset a spin-off multi-lingual newsletter in \XeLaTeX.}

\spacesection{}
% Consider downgrading to the bottom
\section{Technical Skills}
\subsection{Programming Languages}

\cvdoubleitem{\textsc{Experienced}}{C++, Python, R, FORTRAN, Shell (zsh, bash),  OpenMP, OpenMPI, Tcl, CSS, JS, HTML, Sass, C}
{\textsc{Familiar}}{Ruby, Julia, Java, Haskell, Matlab, Golang, ReactJS, Node, CUDA}

\spacesubsection{}
\subsection{Projects}

\cvdoubleitem{\textsc{Experienced}}{d-SEAMS, Android (Cyanogen, LineageOS, AOSP), Web-Design (static), ArchLinux}
{\textsc{Familiar}}{Linux Kernel (Android)}

\spacesubsection{}
\subsection{Simulation Projects}

\cvdoubleitem{\textsc{Experienced}}{\es \space(Extensible Simulation Package for Research on Soft matter), LAMMPS (Large-scale Atomic/Molecular Massively Parallel Simulator), OVITO, AiiDA (Automated Interactive Infrastructure and Database for Computational Science)}
{\textsc{Familiar}}{OpenFOAM, GROMACS (GROningen MAchine for Chemical Simulations), VMD (Visual Molecular Dynamics)}

\spacesubsection{}
\subsection{Tools}

\cvdoubleitem{\textsc{Experienced}}{\XeLaTeX, pandoc, Git (version control), tmux, ssh, Vim, Sublime Text Editor 3, gnuplot, gadfly, bspwm (tiling window manager), babun, MATLAB (matrix laboratory), Continuous Integration Services (Wercker, Travis CI, Semaphore CI), docker}
{\textsc{Familiar}}{AWS (Amazon Web Services), moltemplate, jekyll, middleman, grunt, gulp, Frameworks (Bourbon, Skeleton, neat) Markup Languages (Textile, HAML, Jade(pug))}

% \spacesubsection{}
% \subsection{Operating Systems}
% \cvdoubleitem{\textsc{Preferred}}{ArchLinux}
% {\textsc{Experienced}}{Windows (95, 2000, XP, 7, 8, 10), MacOS (10.7, 10.11, 10.12), Android (1.5, 1.6, 2.2.*, 2.3.*, 4.0.*, 4.4.*, 5.0.*, 6.0.*, 7.* ), Linux Distros (Ubuntu, Sabyon, Puppy, Manjaro, Debian, Red Hat)}

\spacesubsection{}
\subsection{Opensource Contributions}

\cvdoubleitem{\textsc{Created}}{PixN ROM \& Kernel (AOSP based rom for the Xperia Z5) \\ HaoZeke's LineageOS }
{\textsc{Mantained}}{F2PY (NumPy) \\ Xperia Z5 LineageOS (14.*)}

\subsection{Opensource Projects Created}

\cvitem{\textsc{d-SEAMS}}{An open-source, community supported engine for the analysis of molecular dynamics trajectories (co-creator)}

\cvitem{\textsc{wailord}}{FOSS python library to interact with the ORCA suite of quantum chemistry programs, focused on the generation of property specific machine learning data-sets}

\cvdoubleitem{\textsc{zenYoda}}{Pandoc based, tup driven stand-alone multi format (revealJS, beamer etc.) presentation system with static site generation.}
{\textsc{docuYoda}}{A document generation system based on pandoc and latexmk driven by gulp with yaml configuration and easy templating.}

\cvdoubleitem{\textsc{starDock}}{Docker compose based containerized self-updating setup for media hosting, with traefik for reverse proxying. Includes music, ebook and video acquisition and management. \\}
{\textsc{pyQtNumSim}}{A Qt interface for verbose numerical methods assignments.}

\cvdoubleitem{\textsc{grimoire}}{Metalsmith and webpack based open source educational experiment with a strong focus on readability, equations and references. \\}
{\textsc{rgoswami.me}}{A hugo-blog template meant to be used with Emacs orgmode}

% \spacesection{}
% \section{Interests}
% \subsection{Chemical Engineering}
% \cvdoubleitem{\textsc{Experienced}}{Thermodynamics, Transport Phenomena, Mass Transfer, Heat Transfer, Molecular Dynamics (simulations)}
%            {\textsc{Interested}}{Chemical Reaction Engineering (Statistical Interpretation), Process Control}

% \spacesubsection{}
% \subsection{Physics}
% \cvdoubleitem{\textsc{Familiar}}{Statistical Thermodynamics, Density Functional Theory, Rare event sampling, Transition Path Theory, Markov State Models}
%            {\textsc{Interested}}{Quantum Phenomena (Computing, Thermodynamics), Phase Transitions (Thermodynamics, Simulations), Chaos Theory, Spectroscopy, Entropy, Information Theory}

\spacesection{}
\section{Affiliations \& Accolades}

\spacesubsection{}
\subsection{Memberships}

\cventry{\textsc{2014--present}}{OSA (Optical Society of America)}{}{Student Member $\to$ Early Career Member (2018)}{}{}
\cventry{\textsc{2015--present}}{AIChE (American Institute Of Chemical Engineers)}{}{Student Member $\to$ Young Professional (2018)}{}{}
\cventry{\textsc{2015--present}}{APS (American Physical Society)}{}{Student Undergraduate Member $\to$ Early Career Member (2019)}{}{}
\cventry{\textsc{2015--present}}{IEEE (Institute of Electrical and Electronics Engineers)}{}{Student Member $\to$ Early Career Member (2018)}{}{}
\cventry{\textsc{2015--present}}{IOP (Institute of Physics)}{}{Student Member (2018) $\to$ Member (2019)}{}{}
\cventry{\textsc{2006--present}}{World Taekwondo}{}{Red Belt}{}{}
\cventry{\textsc{2009--present}}{XDA Developers}{}{Senior Member}{}{}
% \cventry{\textsc{2019--present}}{SIGHPC (Special Interest Group for High Performance Computing) ACM Chapter}{Professional Member}{}{}{}
% \cventry{\textsc{2019--present}}{SIGHPC-Education ACM Chapter}{Professional Member}{}{}{}
\cventry{\textsc{2019--present}}{AAAI (Association for the Advancement of Artificial Intelligence)}{}{Professional Member}{}{}
\cventry{\textsc{2019--present}}{ACM (Association for Computing Machinery)}{}{Professional Member}{}{Also part of the SIGHPC (Special Interest Group for High Performance Computing) \& SIGHPC-Education}
\cventry{\textsc{2019--present}}{ASAPBio (Accelerating Science and Publication in biology)}{}{Ambassador}{}{}
\cventry{\textsc{2019--present}}{IChemE (Institute of Chemical Engineers)}{}{Associate Member}{}{}
\cventry{\textsc{2019--present}}{IIChE (Indian Institute of Chemical Engineers)}{}{Life Associate Member}{}{}
\cventry{\textsc{2019--present}}{IEEE IAS (Industrial Applications Society)}{}{Member}{}{}
\cventry{\textsc{2019--present}}{IEI (The Institution of Engineers [India])}{}{Associate Member}{}{}
\cventry{\textsc{2019--present}}{InRaSS (Indian Radio Science Society)}{}{Student Member}{}{}
\cventry{\textsc{2019--present}}{OSI (Open Source Initiative)}{}{Individual Member}{}{}
\cventry{\textsc{2019--present}}{OSI (Optical Society of India)}{}{Life Fellow}{}{}
\cventry{\textsc{2019--present}}{SPIE (Society of Photo-Optical Instrumentation Engineers)}{}{Early Career Professional}{}{}
\cventry{\textsc{2019--present}}{The Carpentries}{}{Certified Instructor}{}{}
\cventry{\textsc{2019--present}}{URSI (Union Radio-Scientifique Internationale)}{}{Corresponding Member}{}{}
\cventry{\textsc{2020--present}}{Heterodox Academy}{}{Graduate Affiliate Member}{}{}
\cventry{\textsc{2020--present}}{IGDORE (Institute for Globally Distributed Open Research and Education)}{}{Researcher-in-training}{}{}
\cventry{\textsc{2020--present}}{COS (Center for Open Sciences)}{}{Ambassador}{}{}
\cventry{\textsc{2020--present}}{ims (Institute of Mathematical Statistics)}{}{Student Member}{}{}
\cventry{\textsc{2020--present}}{Bernoulli Society}{}{Student Member}{}{}
\cventry{\textsc{2020--present}}{RSS (Royal Statistical Society)}{}{E-Student Member}{}{}
\cventry{\textsc{2020--present}}{Computing at School}{}{Member}{}{}
\cventry{\textsc{2021--present}}{TeX Users Group}{}{Member}{}{}
\cventry{\textsc{2021--present}}{Nordic Research Software Engineers RY}{}{Member and Secretary (2021)}{}{}
\cventry{\textsc{2021--2025}}{IOP SIG on Computational Physics}{}{Ordinary Member}{}{}

\spacesubsection{}
\subsection{Awards}

\cventry{\textsc{Spring 2021}}{Tensor Methods and Emerging Applications to the Physical and Data Sciences}{Institute for Pure and Applied Mathematics }{IPAM Fellow}{}{}
\cventry{\textsc{December 2016}}{Photonics-2016}{Indian Institute Of Technology Kanpur}{Springer Best Student Paper Award}{Nonlinear-Optics Session}{}
\cventry{\textsc{2014--2015}}{IITG Zephyr Creative Writing}{Indian Institute Of Technology Guwahati}{First Prize}{}{}
\cventry{\textsc{2014--2015}}{Antaragni IITK-MUN GA-DISEC}{Indian Institute Of Technology Kanpur}{Best Speaker}{}{}

\spacesubsection{}
\subsection{Reviews}
\cventry{\textsc{2022--present}}{IOP - Journal of Physics B: Atomic, Molecular and Optical Physics}{Reviewer}{}{}{}
\cventry{\textsc{2022--present}}{IOP - New Journal of Physics}{Reviewer}{}{}{}
\cventry{\textsc{2021--present}}{IOP - Journal of Optics}{Reviewer}{}{}{}
\cventry{\textsc{2021--present}}{IOP - Journal of Optics}{Reviewer}{}{}{}
\cventry{\textsc{2021--present}}{IOP - Journal of Physics: Condensed Matter}{Reviewer}{}{}{}
\cventry{\textsc{2021--present}}{APS Physical Review A}{Reviewer}{}{}{}
\cventry{\textsc{2021--present}}{APS Physical Review E}{Reviewer}{}{}{}
\cventry{\textsc{2021--present}}{APS Physical Review Applied}{Reviewer}{}{}{}
\cventry{\textsc{2020--present}}{Nature Communications}{Reviewer}{}{}{}
\cventry{\textsc{2020--present}}{APS Physical Review Letters}{Reviewer}{}{}{}
\cventry{\textsc{2020--present}}{IOP - Machine Learning: Science and Technology}{Reviewer}{}{}{}
\cventry{\textsc{2020--present}}{SPIE Journal of Micro/Nanolithography, MEMS, and MOEMS}{Reviewer}{}{}{}
\cventry{\textsc{2020--present}}{Manning Publications}{Technical Reviewer}{}{}{}
\cventry{\textsc{2020--present}}{JupyterCon 2020}{Reviewer}{}{}{}
\cventry{\textsc{2020--present}}{Nature Communications}{Reviewer}{}{}{}
\cventry{\textsc{2019--present}}{PeerJ - Life \& Environment}{Reviewer}{}{}{}
\cventry{\textsc{2019--present}}{PeerJ - Computer Science}{Reviewer}{}{}{}
\cventry{\textsc{2019--present}}{PeerJ - Organic Chemistry}{Reviewer}{}{}{}
\cventry{\textsc{2019--present}}{PLOS ONE}{Reviewer}{}{}{}
\cventry{\textsc{2018--present}}{Journal Of Open Source Software}{Reviewer}{}{}{}
  % I review submissions pertaining to molecular dynamics,virtualization, HPC, web platforms, finite element methods, optimization and computer geometry written primarily in C++, Rust, FORTRAN, Julia, Javascript. I am also listed for Python and R submissions.
  % Specifically:
	% \begin{itemize}
	% 	\item \href{https://joss.theoj.org/papers/76800aa00f4d57b1695c876c0b936ba3}{DEPP - Differential Evolution Parallel Program}
	% 	\item \href{https://joss.theoj.org/papers/553250d815e1990db1b89c742854c71a}{RHEOS - A Julia package for Rheology Data Analysis}
	% 	\item \href{https://joss.theoj.org/papers/25d51faf73cc17ae3affb51b787bbe18}{Computing diffusion coefficients in macromolecular simulations: the Diffusion Coefficient Tool for VMD}
	% 	\item \href{http://joss.theoj.org/papers/093a37e7d698d3350b2d4e564dcfb69b}{Simple-Web-Server: a fast and flexible HTTP/1.1 C++ client and server library}
	% 	\item \href{http://joss.theoj.org/papers/033b265ce2f8562dc1148b534a09275c}{HyperNaut: a navigator for the hyperbolic plane}
	% 	\item \href{http://joss.theoj.org/papers/fc7f571e455ef955d1e5dd271343723f}{The Biddy BDD package}
	% 	\item \href{http://joss.theoj.org/papers/f3b70136a43eb28b9327ec1ce42533c2}{Prest: Open-Source Software for Computational Revealed Preference Analysis}
	% \end{itemize}
% \cventry{\textsc{2021--present}}{APS Physical Review E}{Reviewer}{}{}{}

\spacesection{}
\section{Grants Awarded}
\cventry{\textsc{2020--2023}}{Icelandic Research Fund}{Rannís}{6.650 thousand ISK}{Doctoral Fellowship}{\textsc{Title: }Magnetic interactions of itinerant electrons modeled using Bayesian machine learning.}

\spacesection{}
\section{Publications}
 {\color{color1}\textsc{Journals}}

\nocite{*}
\printbibliography[heading=none, env=midbib, keyword=journal]

{\color{color1}\textsc{Conference Proceedings}}

\nocite{*}
\printbibliography[heading=none, env=midbib, keyword=conference]

% Currently none which haven't been published
% {\color{color1}\textsc{Preprints}}

% \nocite{*}
% \printbibliography[heading=none, env=midbib, keyword=preprint]

\spacesection{}
\section{Conference Records}

\subsection{Posters}
\cventry{\textsc{July 2022}}{Wailord: Parsers and Reproducibility for Quantum Chemistry}{SciPyCon 2022}{R. Goswami}{}{}
\cventry{\textsc{July 2022}}{LPython: Interactive LLVM-based Python Compiler for Modern Architectures}{SciPyCon 2022}{O. Čertík, B. Beckman, N. Gera, S. Lunagariya, G. Singh, R. Goswami, T. Shaktivel, and D. Edwards}{}{}
\cventry{\textsc{March 2020}}{Ultrafast Insights for Predictive Fragrance Compounding}{ACS Spring 2020 National Meeting}{R. Goswami, A. K. Rawat, D. Chakrabarty, and D. Goswami}{}{}
\cventry{\textsc{December 2019}}{Qubit Network Barriers to Deep Learning}{IEEE WRAP-2019}{R. Goswami, A. Goswami, and D. Goswami}{}{}
\cventry{\textsc{March 2019}}{Space Filling Curves: Heuristics For Semi Classical Lasing Computations}{URSI Asia-Pacific Radio Science Conference (AP-RASC 2019)}{R. Goswami, A. Goswami, and D. Goswami}{}{}
\cventry{\textsc{December 2018}}{FDTD Numerical Computations for Ultrafast Non-linear Optics}{Photonics-2018}{R. Goswami, A. Goswami, and D. Goswami}{}{}

\subsection{Oral Presentations}
\cventry{\textsc{July 2022}}{Maintaining Fortran in Python in Perpetuity}{SciPyCon 2022}{R. Goswami, M. Mendonca, R. Gommers, and T. Shaktivel}{}{}
\cventry{\textsc{September 2021}}{f2py: Two Decades Later}{FortranCon 2021}{R. Goswami, R. Gommers, M. Mendonca, and P. Peterson}{}{}
\cventry{\textsc{September 2021}}{Implementing Fortran Standardese within LFortran}{FortranCon 2021}{R. Goswami and O. Čertík}{}{}
\cventry{\textsc{August 2021}}{Continuous Integration and TeX with Org-Mode}{TUG21, 42nd Annual Conference of the TeX Users Group}{R. Goswami}{}{}
\cventry{\textsc{July 2021}}{Modern documentation across languages}{ACM SERI 2021}{R. Goswami}{}{}
\cventry{\textsc{October 2020}}{Nix from the dark ages (without Root)}{NixCon 2020}{R. Goswami}{Lightning Talk}{}
\cventry{\textsc{October 2020}}{Reproducible Scalable Workflows with Nix, Papermill and Renku}{PyCon India 2020}{R. Goswami}{}{}
\cventry{\textsc{August 2020}}{Reproducible Environments with the Nix Packaging System}{CarpentryCon 2020}{R. Goswami and A. Goswami}{}{}
\cventry{\textsc{July 2020}}{LFortran: Interactive LLVM-based Fortran Compiler for Modern Architectures}{FortranCon 2020}{O. Čertík, N. Maan, A. Pandey, M. Curcic, P. Brady, Z. Jibben, N. Carlson, R. Goswami, A. Shahmoradi and A. Markus}{Presented by Ondřej}{}
\cventry{\textsc{December 2019}}{Process Safety in terms of Latent Dirichlet Allocations}{72nd Annual Session of of the Indian Institute of Chemical Engineers, CHEMCON-2019}{R. Goswami}{Accepted}{}
\cventry{\textsc{October 2019}}{Semi-Supervised Approaches to Ultrafast Optimal Control Theory}{43rd Symposium of the Optical Society of India, International Conference on Optics \& Electro-Optics}{R. Goswami, A. Goswami and D. Goswami}{}{}
\cventry{\textsc{December 2016}}{Quantum Distributed Computing with Shaped Laser Pulses}{13th International Conference on Fiber Optics and Photonics}{R. Goswami and D. Goswami}{}{}

\subsection{Sprint Mentorship}
\cventry{\textsc{October 30\textsuperscript{th} 2020}}{NumPy-SciPy Sprint}{PyData Global 2021}{M. Mendonca, M. Picus, R. Goswami, R. Gommers and M. Pahari}{}{}
\cventry{\textsc{September 20\textsuperscript{th} 2020}}{NumPy Sprint}{PyCon India 2021}{M. Mendonca, R. Goswami and M. Pahari}{}{}

\spacesection{}
\section{Workshops and Schools}
\cventry{\textsc{October 2021}}{PRACE Autumn School 2021}{CSC-IT Center for Science, Finland}{This 5-day course held in Vuokatti, consists of lectures and hands-on training on modern, GPU-accelerated high-performance computing: GPU programming and GPU code optimization at scale, as well as understanding and applying machine learning methods emphasizing usage on LUMI}{}{}
\cventry{\textsc{August-September 2021}}{Multiple scattering Green's function for electronic structure and spectroscopy calculations}{Les Houches Physics School, France}{Two week school covering mathematical foundations and applications using Green's function and T-matrix approaches for condensed matter}{}{}
\cventry{\textsc{August 2021}}{Seventh LAMMPS Workshop and Symposium}{Online}{Three day tutorial and overview of recent results led by the LAMMPS developers}{}{}
\cventry{\textsc{July 2021}}{AiiDA Virtual Tutorial}{EPFL, Lausanne, Switzerland}{A five day online in-depth tutorial by core and plugin developers covering programming practices including a hackathon on workflows and plugins}{}{}
\cventry{\textsc{June 2021}}{Comscope Summer School}{Virtual, Comscope, US}{Covers dynamical mean field theory and electronic structure codes using FlapwMBPT, LDA+, LQSGW+DMFT, RISB+LDA by tutorials and training from professionals at BNL and Rutgers}{}{}
\cventry{\textsc{June 2021}}{Virtual Summer School on Theoretical Methods for Energy Conversion}{International Center for Advanced Studies of Energy Conversion (ICASEC) and the RTG2455 BENCh}{The Summer School provided a comprehensive picture of state-of-the-art theoretical methods in the field, combining tutorial and research lectures into a 3-day event. The topics covered spanned electronic structure methods, dynamics and machine learning approaches}{}{}
\cventry{\textsc{June 2021}}{MPI and OpenMP in Scientific Software Development}{PRACE @ SURF, NL}{An in-depth application oriented three day event with concrete examples in C++ involving guest lectures from distinguished speakers and covering parallel I/O, debugging performance bottlenecks, multi-grid and mesh methods, along with more of the OpenMPI standard; culminated with examples of using OpenMP for GPUs; from the PRACE training center at SURF}{}{}
\cventry{\textsc{June 2021}}{Basic Parallel Programming with MPI and OpenMP}{SURF, NL}{A refresher on OpenMPI and OpenMP for both Fortran and C++ usage, involving the HLRI materials}{}{}
\cventry{\textsc{May 2021}}{Efficient Tensor Representations for Learning and Computational Complexity}{Institute for Pure and Applied Mathematics, US}{The fourth of the core IPAM long program workshops; focusing on algorithmic complexity and usage}{}{}
\cventry{\textsc{May 2021}}{Mathematical Foundations and Algorithms for Tensor Computations}{Institute for Pure and Applied Mathematics, US}{The third of the core IPAM long program workshops; focusing on numerical analysis and linear algebra for tensor networks and their computation}{}{}
\cventry{\textsc{April 2021}}{Tensor Network States and Applications}{Institute for Pure and Applied Mathematics, US}{The second of the core IPAM long program workshops; focusing on tensor network states; their topology and applications}{}{}
\cventry{\textsc{March-April 2021}}{Tensor Methods and their Applications in the Physical and Data Sciences}{Institute for Pure and Applied Mathematics, US}{The first of the workshops which form the core of the IPAM long program; focusing on representations of basic tensor networks in various disciplines and their computation}{}{}
\cventry{\textsc{March 2021}}{SWiMM: School on Simulation Workflows in Materials Modelling}{CECAM School}{Two week long program focused on the familiarization with tools for high-throughput HPC workflows including AiiDA and PyIron along with tools ontological exploration.}{}{}
\cventry{\textsc{March 2021}}{Workflows for Atomistic Simulations}{Ruhr-Universität Bochum}{A three day intensive workshop consisting of hands-on sessions working with PyIron followed by lectures and discussions on potential and forcefield development}{}{}
\cventry{\textsc{March 2021}}{Tensor Method Tutorials}{Institute for Pure and Applied Mathematics, US}{Part of my long program participation in the Tensor Methods and Emerging Applications to the Physical and Data Sciences. A two week long exposure to the basics of the program, and recent advances for context.}{}{}
\cventry{\textsc{January 2021}}{IPWin2021: Inverse problems in partial differential equations and geometry}{Technical University of Denmark}{Virtual winter school providing an introduction to modern and hot topics in inverse problems based on three series of embedded lectures given by internationally recognized researchers}{}{}
\cventry{\textsc{January 2021}}{BerkeleyGW Workshop and Berkeley Excited States Conference}{Virtual}{An event focused on the BerkelyGW software, split between a three day practical tutorial and a two day conference}{}{}
\cventry{\textsc{October 2020}}{Computational materials discovery of unconventional magnets}{EPFL, Lausanne, Switzerland}{The proposed workshop aims to bring together researchers both from theory and experiments working on different aspects of the field to present the latest advancements and to discuss the efforts required to expedite the discovery of new unconventional magnetic materials}{}{}
\cventry{\textsc{October 2020}}{ESPResSo and Python: Versatile Tools for Soft Matter Research}{CECAM School}{The school focuses on the introduction of particle-based coarse-grained Molecular Dynamics simulation techniques for Hard and Soft Matter systems with the freely available software package \es. Includes lectures and hands-on sessions}{}{}
\cventry{\textsc{September 2020}}{Excited Charge Dynamics in Semiconductors}{ICTP, Trieste}{An online workshop to discuss recent progress in the investigation of dynamics of excited charges in semiconductors.}{}{}
\cventry{\textsc{September 2020}}{(Machine) learning how to coarse-grain}{CECAM and TRR146 Virtual Event}{The purpose of this workshop will be to discuss the current state of the art, some of the challenges that the community is facing in furthering the penetration of ML models in CG simulations, and future perspectives}{}{}
\cventry{\textsc{September 2020}}{Gaussian Process and Uncertainty Quantification Summer School}{University of Sheffield, UK}{The Gaussian Process Summer Schools are a series of schools and workshops aimed at researchers who want to understand and use Gaussian process models, both in theory and practice}{}{}
\cventry{\textsc{July 2020}}{AiiDA Virtual Tutorial}{EPFL, Lausanne, Switzerland}{A four day online in-depth tutorial by core and plugin developers covering the nuances of reproducible workflows for AiiDA with example Quantum ESPRESSO calculations on Quantum Mobile virtual machines}{}{}
\cventry{\textsc{June 2020}}{Mathematical Methods of Modern Statistics 2}{CIRM Virtual Event}{A five day online event from the Centre International de Rencontres Mathématiques on recent statistical advances relating to the analysis of high dimensional data from frequentist and Bayesian perspectives}{}{}
\cventry{\textsc{May 2020}}{ALCF Computational Performance Workshop}{Online}{A three day workshop on the hardware and software improvements organized by Argonne Leadership Computing Facility}{}{}
\cventry{\textsc{October 2019}}{TriangleSCI 2019}{Invited to the Triangle Scholarly
	Communication Institute}{A week long fully-funded incubator to discuss
	actionable goals towards Bringing Equity and Diversity to Peer Review. This
	was undertaken as part of the larger discussion on Equity in Scholarly
	Communications}{\textbf{declined to attend}}{}
\cventry{\textsc{May-June 2019}}{Artificial Intelligence}{E \& ICT Academy, IIT Kanpur}{A four week course on AI foundations culminating in a time-series prediction project}{}{}
\cventry{\textsc{June 2019}}{AI Foundations Certificate Course}{univ.ai}{A summer school taught in-person by faculty from Harvard and UCLA, culminating in a computer vision and neural network based identification project}{}{}
\cventry{\textsc{July 2019}}{Rare Events Summer School}{Indian Institute of Science, Bangalore}{A short course consisting of lectures and hands-on sessions by experts in the field, organized by Prof. Baron Peters}{}{}
\subsection{Short Courses}
\cventry{\textsc{September 23\textsuperscript{rd} 2019}}{Surface Area and Porous Material Characterization}{Dept. of ChemE, IIT Kanpur}{An intensive day long course on the basics of experimental classification and DFT methods for pore distribution by Dr. Martin Thomas from Anton-Paar}{}{}
\cventry{\textsc{September 21\textsuperscript{st} 2019}}{OpenACC GPU Bootcamp}{Chemistry Department, IIT Kanpur}{Day long programming session and discussion covering the acceleration of Institute in-house code facilitated by a Senior Nvidia Solution Architect (Mr. Bharatkumar) and Prof. Debabrata Goswami}{}{}

% TODO: Update with IPWin2021 maybe

\spacesection{}
\section{Certifications}

\subsection{Coursera}
\cventry{\textsc{March 2020}}{Deep Learning
  Specialization}{deeplearning.ai}{}{$99.5\%$}{
  \href{https://www.coursera.org/account/accomplishments/specialization/Q3DLMMZJ42TR}{The specialization can be verified by ID}
  \href{https://www.coursera.org/account/accomplishments/specialization/Q3DLMMZJ42TR}{\ttfamily
	Q3DLMMZJ42TR}. This signifies the completion of the following five courses:
	\begin{itemize}
		\item
		\href{https://www.coursera.org/account/accomplishments/verify/WFMW5VPBCKVJ}{Neural
		Networks and Deep Learning} - $\mathbf{100.0\%}$
		\item
		\href{https://www.coursera.org/account/accomplishments/records/9NJJAPSMK59A}{Improving
		Deep Neural Networks: Hyperparameter tuning, Regularization and
		Optimization} - $\mathbf{99.2\%}$
		\item
		\href{https://www.coursera.org/account/accomplishments/records/QWTW54L3CEDC}{Structuring
		Machine Learning Projects} - $\mathbf{98.3\%}$
		\item
		\href{https://www.coursera.org/account/accomplishments/verify/A23X8Q7QHJPV}{Convolutional
		Neural Networks} - $\mathbf{100.0\%}$
		\item
		\href{https://www.coursera.org/account/accomplishments/records/D8JCTU57D6CN}{Sequence
		Models} - $\mathbf{100.0\%}$
	\end{itemize}
}

% Consider moving to be below Education
\spacesubsection{}
\subsection{NPTEL Courses}
\cventry{\textsc{Jan-Apr 2019}}{Graph Theory}{IISER Pune}{}{$55\%$}{\href{https://nptel.ac.in/noc/E_Certificate/noc19-ma13/NPTEL19MA13S21460067191106132.jpg}{License: NPTEL19MA13S21460067}}
\cventry{\textsc{Aug-Sep 2018}}{Computational Chemistry and Classical Molecular Dynamics}{IIT Bombay}{Elite}{$77\%$}{\href{https://nptel.ac.in/noc/social_cert/noc18-cy13/NPTEL18CY13S214401271810100922.jpg}{License: NPTEL18CS13S21440127}}
\cventry{\textsc{Aug-Sep 2018}}{Introduction to Parallel Programming in OpenMP}{IIT Madras}{}{$40\%$}{\href{https://nptel.ac.in/noc/social_cert/noc18-cs55/NPTEL18CS55S114401221810100922.jpg}{License: NPTEL18CS55S11440122}}
\cventry{\textsc{Jan-Apr 2018}}{Quantum Computing}{IIT Kanpur}{Elite}{$65\%$}{\href{https://nptel.ac.in/noc/social_cert/noc18-cy07/NPTEL18CY07S44800241810011238.jpg}{License: NPTEL18CY07S4480024}}

\spacesection{}
\section{Graduate Coursework}

\spacesubsection{}
\blockquote[Note]{Due to COVID-19, grading varies from Pass/Fail to a 10 point scale.}\\
\spacesubsection{}

\cventry{\textsc{Summer 2020}}{Seminar on Machine Learning}{University of Iceland}{Computer Science Dept.}{Pass}{Research seminar.\\ \textsc{Presentation:} Graph Neural Networks}
\cventry{\textsc{Summer 2020}}{Statistics, science and
	COVID-19}{University of Iceland}{Health Sciences
	Institute}{$10$}{Course covering ``Statistical Rethinking'' by Richard McElreath as well as the COVID-19 model of the University.}
\cventry{\textsc{Spring 2020}}{Probabilistic Data Analysis}{University of Turku, Finland}{Turku Data Science group}{}{Research seminar based on ``Bayesian Data Analysis'' by Andrew Gelman, John Carlin, Hal Stern, David Dunson, Aki Vehtari, and Donald Rubin\\ \textsc{Presentation:} Markov Chain Simulations}
\cventry{\textsc{Spring 2020}}{Applied data analysis}{University of Iceland}{Mathematics and Applied Statistics Dept.}{$10$}{Course covering ``Introduction to Statistical Learning An Introduction to Statistical Learning with Applications in R'' by Gareth James, Daniela Witten, Trevor Hastie and Robert Tibshirani\\ \textsc{Project:} Icelandic Housing Analysis}
\cventry{\textsc{Spring 2020}}{Applied data analysis - Project}{University of Iceland}{Mathematics and Applied Statistics Dept.}{Pass}{Extra credit course\\ \textsc{Project:} Molecular Dynamics Trajectory Analysis}
\cventry{\textsc{Spring 2020}}{Machine Learning}{University of Iceland}{Computer Science Dept.}{Pass}{An overview of some of the main concepts, techniques and algorithms in machine learning. Supervised learning, unsupervised learning and reinforcement learning. Data preprocessing and data visualization. Model evaluation and model selection. Linear regression, nearest neighbors, support vector machines and decision trees. Deep learning. Cluster analysis, the k-means and EM algorithms. TD-learning, Q-learning. \\ \textsc{Project:} Toxic Comment Classification (Grade $9/10$)}
\cventry{\textsc{Spring 2020}}{Software Quality Management}{University of Iceland}{Pass}{}{Metrics and models: Tools for managing software quality and its improvement.Quality Assurance: standards. Software Quality assurance activities. Quality assurance infrastructure and progress control. Software quality in Agile development.\\ \textsc{Presentation:} Scrum (Grade $10/10$)}
\cventry{\textsc{Spring 2020}}{Introduction to Pattern Recognition}{University of Iceland}{Electrical Engineering Dept.}{$8$}{Course covering ``Elements of Statistical Learning: Data Mining, Inference, and Prediction'' by Trevor Hastie, Robert Tibshirani, and Jerome Friedman along with ``Pattern Recognition and Machine Learning
'' by Christopher Bishop. \\ \textsc{Project:} Predicting Molecular Properties}
% \cventry{\textsc{Spring 2020}}{Machine Learning}{University of Iceland}{An overview of some of the main concepts, techniques and algorithms in machine learning. Supervised learning, unsupervised learning and reinforcement learning. Data preprocessing and data visualization. Model evaluation and model selection. Linear regression, nearest neighbors, support vector machines and decision trees. Deep learning. Cluster analysis, the k-means and EM algorithms. TD-learning, Q-learning. \\ \textbf{Project:} Toxic Comment Classification (Grade 9/10)}{}{}
% \cventry{\textsc{Spring 2020}}{Applied data analysis}{University of Iceland}{Course covering ``Introduction to Statistical Learning An Introduction to Statistical Learning with Applications in R'' by Gareth James, Daniela Witten, Trevor Hastie and Robert Tibshirani\\ \textbf{Project:} Toxic Comment Classification}{$10$}{}

%{\color{color1}\textsc{Fictional Writings}}
%\printbibliography[heading=none, env=midbib, keyword=fiction]

%=============================
% this part is a simple cover letter
%\clearpage

%\recipient{Human Resources}{Some Company Ltd.\\Some Street 123\\12345 Some City}
%\date{\today}
%\opening{Dear Sir or Madam,}
%\closing{Sincerely yours,}
%\enclosure[Attached]{curriculum vit\ae{}}

%\makelettertitle

%\lipsum[1-3]

%\makeletterclosing


\end{document}

%%% Local Variables:
%%% mode: latex
%%% TeX-master: t
%%% End:
