\documentclass[11pt,a4paper, final, factor=1100, stretch=18, shrink=18]{moderncv}

\usepackage{color}
\usepackage{fontspec}

% Stop bugging me about \es
\usepackage{xspace}
% moderncv themes
\moderncvstyle{classic}                             % style options are 'casual' (default), 'classic', 'banking', 'oldstyle' and 'fancy'
%\moderncvcolor{blue}                               % color options 'black', 'blue' (default), 'burgundy', 'green', 'grey', 'orange', 'purple' and 'red'

% the (custom) color which will be used in the cv
\definecolor{color1}{RGB}{1, 52, 64}

% scale the page layout (depreciated for more complicated stuff)
%\usepackage[scale=0.75]{geometry}

% change width of the column with the dates
\setlength{\hintscolumnwidth}{2.5cm} 

%\PassOptionsToPackage[final, factor=1100, stretch=18, shrink=18 ]{microtype} 	
% The final option overrides global defaults. It greatly improves general appearance of the text. The stretch and shrink reduce bluriness[20,20 default]. The factor increases protrusion amount by 10%, [default 1000]
% Tracking allows for small caps, like in cites to be adjusted. The activate commands are to set protrusion.

\microtypecontext{spacing=nonfrench} 						% To preserve interword spacing via \nonfrenchspacing.
 
\SetExtraKerning[unit=space] 								% These produce more effects, from microtype with kerning.
    {encoding={*}, family={bch}, series={*}, size={footnotesize,small,normalsize}}
    {\textendash={400,400}, 								% en-dash, add more space around it
     "28={ ,150}, 											% Left bracket, add space from right
     "29={150, },											% Right bracket, add space from left
     \textquotedblleft={ ,150}, 							% Left quotation mark, space from right
     \textquotedblright={150, }} 							% Right quotation mark, space from left


\SetExtraKerning[unit=space]
  {encoding={*}, family={qhv}, series={b}, size={large,Large}}
  {1={-200,-200}, 
   \textendash={400,400}}

\SetTracking{encoding={*}, shape=sc}{40} 					% This is for better small caps with tracking and microtype.

\SetProtrusion{encoding={*},family={bch},series={*},size={6,7}}  % This enables better optical views.
              {1={ ,750},2={ ,500},3={ ,500},4={ ,500},5={ ,500},
               6={ ,500},7={ ,600},8={ ,500},9={ ,500},0={ ,500}}

% Page settings
\usepackage{geometry}
 \geometry{
 a4paper,
 %total={210mm,297mm},
 left=15mm,
 right=15mm,
 top=15mm,
 bottom=10mm,
 }

% required when changing page layout lengths
\AtBeginDocument{\recomputelengths} 
\usepackage{xunicode}
\usepackage{xltxtra}
% \usepackage[utf8]{inputenc} (Xelatex expects UTF8 anyway)

% I like pretty logos
\usepackage{metalogo}

% Widen the letter spacing for logos
\setlogokern{La}{0.2pt}
\setlogokern{Xe}{0.2pt}
\setlogokern{eL}{0.2pt}
\setlogokern{Te}{0.2pt}
\setlogokern{aT}{0.2pt}
\setlogokern{eX}{0.2pt}

% Better Quotes (depreciated for this useage)
%\usepackage{epigraph}

% Poaching from Espresso's ug.tex

%%%%%%%%%%%%%%%%%%%%%%%%%%%%%%%%%%%%%%%%%%%%%%%%%%
%%%%%%%%%%%%%%%%%%%%%%%%%%%%%%%%%%%%%%%%%%%%%%%%%%
%%%%%%%%% New Commands and Environments %%%%%%%%%%
%%%%%%%%%%%%%%%%%%%%%%%%%%%%%%%%%%%%%%%%%%%%%%%%%%
%%%%%%%%%%%%%%%%%%%%%%%%%%%%%%%%%%%%%%%%%%%%%%%%%%
\newcommand{\es}{\mbox{\textsf{ESPResSo}}\xspace}

% german word break/hyphenation rules that's with ngerman (we use english)
\usepackage[british,UKenglish,USenglish,american]{babel}

% insert dummy text (used in the letter)
%\usepackage{lipsum} 

% used for \begin{comment}...\end{comment}
\usepackage{verbatim} 

% use guilllemets in bibliography (not german.) Also autostyle superceeded babel
\usepackage[autostyle=true]{csquotes}

\usepackage[
	sorting=none,
	minbibnames=8,
	maxbibnames=9,
	block=space
]{biblatex}

\bibliography{publications}


% get rid of the number-labels ([1], [2], etc.) in front of publications
\defbibenvironment{midbib}
{\list
	{}
	{
		\setlength{\leftmargin}{0mm}
		\setlength{\itemindent}{-\leftmargin}
		\setlength{\itemsep}{\bibitemsep}
		\setlength{\parsep}{\bibparsep}}
	}
	{\endlist}
{\item}


% add [DOI] and [PDF] fields at the end of each publication entry
\DeclareSourcemap{
	\maps[datatype=bibtex]{
		% the bibtex entry 'mydoi' gets mapped to 'usera'
		\map{
			\step[fieldsource=mydoi]
			\step[fieldset=usera, origfieldval]
		}
		% the bibtex entry 'mypdf' gets mapped to 'usera'
		\map{
			\step[fieldsource=mypdf]
			\step[fieldset=userb, origfieldval]
		}
	}
}

% [DOI] entries in publication
\DeclareFieldFormat{usera}{\color{color1}[\href{#1}{\textsc{doi}}]}
\AtEveryBibitem{
	% put [DOI] stuff at the end of a publication entry
	\csappto{blx@bbx@\thefield{entrytype}}{% 
		\iffieldundef{usera}{ 
			% this gets invoked, once nothing is supplied
			% via the mypdf or mydoi value.
			% you could e.g. display a default thing here.
		}{\space\printfield{usera}}}
}

% [PDF] entries in publication
\DeclareFieldFormat{userb}{\color{color1}[\href{#1}{\textsc{pdf}}]}

\AtEveryBibitem{
	% put [DOI] stuff at the end of a publication entry
	\csappto{blx@bbx@\thefield{entrytype}}{\iffieldundef{userb}{}{\printfield{userb}}}
}

\renewcommand*{\mkbibnamegiven}[1]{%
	\ifitemannotation{highlight}
	{\textbf{#1}}
	{#1}}

	\renewcommand*{\mkbibnamefamily}[1]{%
	\ifitemannotation{highlight}
	{\textbf{#1}}
	{#1}
}


% Minion Pro is used as the main font, if you don't
% have it installed uncomment this line or choose another pretty font.
% Literata is also included.
%\setmainfont[Numbers=OldStyle]{Minion Pro}

\setmainfont[
Path = Fonts/MinionPro/,
Numbers=OldStyle,
Extension = .otf,
UprightFont = *_Regular,
ItalicFont = *_It,
BoldFont = *_Bold,
BoldItalicFont = *_BoldIt
]{MinionPro}

% Myriad Pro is used as the sans font, if you don't
% have it installed uncomment this line or choose another pretty font.
\setsansfont[
Path = Fonts/MyriadPro/,
Numbers=OldStyle,
Extension = .otf,
UprightFont = *_Regular,
ItalicFont = *_It,
BoldFont = *_Bold,
BoldItalicFont = *_BoldIt,
Ligatures=TeX,
Scale=MatchLowercase]{MyriadPro}

% the moderncv package will populate a lot of the pdf meta-information.
% this can be used to change some of these infos.
\AfterPreamble{\hypersetup{
  pdfcreator={XeLaTeX},
  pdftitle={First fledging CV}
}}

% for the icons (telephone, globe). I found the icons provided by the
% fontawesome package prettier than the standard moderncv icons.
\defaultfontfeatures{
    Path = Fonts/FontAwesome/ }
\usepackage{fontawesome}

% personal data
\firstname{Rohit}
\familyname{Goswami}
\address{House No. 646, 35th Lane}{IIT Kanpur, 208016}
\quote{``An unproblematic state is a state without creative thought. It's other name is Death.''\\-- David Deutsch}

% \faEnvelope \faPhone \faGithub \faGlobe

% I use the extrainfo command for additional information, since I 
% want to use custom icons and have finer control over spacing.
\extrainfo{
	\faPhone\hspace{0.3em}+91 9935135006\\
	{\small\faEnvelope}\hspace{0.3em}r95g10@gmail.com\\
	{\small\faGithub}\hspace{0.3em}haozeke\\
	\faGlobe\hspace{0.3em}https://github.com/HaoZeke/
}


% picture, resized to a height of 84pt
\photo[84pt]{Picture/rohit}

% spacing before (sub)sections
\newcommand{\spacesection}{\vspace{0.4cm}}
\newcommand{\spacesubsection}{\vspace{0.2cm}}


%===========================
\begin{document}

\maketitle

\section{Personal Data}
\cvitem{Name}{Rohit Goswami}
\cvitem{Date Of Birth}{10.08.1995}
%\cvitem{Geburtsort}{Petrovichi, Russia}

\spacesection
\section{Education}
%\cventry{\textsc{1948}}{Ph.D. Chemistry}{University Extension}{New York}{\emph{United States}}{}
%\cventry{\textsc{1939--1941}}{M.Sc. Chemistry}{University Extension}{New York}{\emph{United States}}{}
\cventry{\textsc{2014--2018}}{B.Tech. Chemical Engineering}{Harcourt Butler Technical University}{Kanpur}{\emph{India}}{$61.94\%$ First Division}
\cventry{\textsc{2011--2013}}{Intermediate (AISSCE)}{Delhi Public School Kalyanpur}{Kanpur}{\emph{India}}{$87.2\%$ Central Board of Secondary Education (CBSE)}
\cventry{\textsc{2009--2011}}{High School (AISSE)}{Delhi Public School Kalyanpur}{Kanpur}{\emph{India}}{$9.8$ Cumulative Grade Point Average (CGPA) in Central Board of Secondary Education (CBSE)}

\spacesection
\section{Experience}
\subsection{Internships}
\cventry{\textsc{2017-Present}}{Dr. Debojit Chakrabarty}{Keva Fragrances Ltd, Mumbai}{}{R\&D Industrial Intern}{Modeling complex multi-component perfumes in a predictive method via experimental and theoretical considerations. In collaboration with Prof. Rajdip Bandyopadhyaya of the ChemE Dept. at IIT Bombay.
}
\cventry{\textsc{Summer 2017}}{Prof. Sibasish Ghosh}{The Institute of Mathematical Sciences, Chennai}{}{Visiting Scholar}{Discussed computational techniques for the simulation and understanding of quantum tomography.}
\cventry{\textsc{Summer 2017}}{Dr. Nisanth Nair}{Indian Institute Of Technology Kanpur}{}{SURGE Scholar}{An exploratory project to understand and deal with bottlenecks in computational chemistry, the major objectives were to investigate hybridization of existing code via OpenMP and MPI.
\\~\\
 \textsc{Poster: }Development of Computational Tools for Free Energy Calculations of Chemical Reactions}
\cventry{\textsc{Summer 2016}}{Dr. Rajarshi Chakrabarti}{Indian Institute Of Technology Bombay}{}{Research Intern}{ Retooled a server with ArchLinux and also simulated patchy colloids (Janus Particles). \\~\\
 \textsc{Project Report: }Computational Survey of Coarse Grained Soft Matter Molecular Dynamics Simulations
}

\subsection{Volunteer Work}

\cventry{\textsc{2017--2018}}{ChemE Herald}{Harcourt Butler Technical University, Kanpur}{}{Editor-in-Chief}{Inaugurated and managed an interdisciplinary technical newsletter.}

\cventry{\textsc{2017--2018}}{HBTU-MUN 2018}{}{}{Secretary General}{Designed a ReactJS based static website, with Trello backed user registration, also performed outreach pre-events to raise awareness and participation, in addition to overseeing the working of the executive board.}

\cventry{\textsc{2016--2017}}{HBTU-MUN 2017}{}{}{Executive Board Chairperson}{Designed a Jekyll based static website and ensured adherence to standard MUN rules as Chairperson.}

% Sadly this was hit by anti-aircraft guns
% \cventry{\textsc{2016--2017}}{Interface 2017}{Harcourt Butler Technical University, Kanpur}{}{Technical Cell Head}{Organizing and inspiring students to work towards the success of the Department's techno-cultural fest and seminars.}

\cventry{\textsc{2014--2016}}{The Curiosity Magazine}{Harcourt Butler Technical University, Kanpur}{}{Editor-in-Chief}{Managed a diverse team of student content writers and also later typeset a spin-off multi-lingual newsletter in \XeLaTeX.}

\spacesection
\spacesubsection
\section{Technical Skills}
\spacesubsection
\subsection{Programming Languages}
\spacesubsection

\cvdoubleitem{\textsc{Experienced}}{CSS, JS, HTML, Sass, C, Tcl}
           {\textsc{Familiar}}{Ruby, C++, Julia, Python, Shell (zsh, bash), Golang, FORTRAN, ReactJS, Node}

\spacesubsection
\subsection{Projects}
\spacesubsection

\cvdoubleitem{\textsc{Experienced}}{Android (Cyanogen, LineageOS, AOSP), Web-Design (static), ArchLinux}
           {\textsc{Familiar}}{Linux Kernel (Android)}

\spacesubsection
\subsection{Simulation Projects}
\spacesubsection

\cvdoubleitem{\textsc{Experienced}}{\es \space(Extensible Simulation Package for Research on Soft matter), LAMMPS (Large-scale Atomic/Molecular Massively Parallel Simulator)}
           {\textsc{Familiar}}{OpenFOAM, GROMACS (GROningen MAchine for Chemical Simulations), VMD (Visual Molecular Dynamics), CPMD (Car-Parrinello Molecular Dynamics)}

\spacesubsection
\subsection{Tools}
\spacesubsection

\cvdoubleitem{\textsc{Experienced}}{\XeLaTeX, pandoc, Git (version control), tmux, ssh, Vim, Sublime Text Editor 3, gnuplot, gadfly, bspwm (tiling window manager), mosh, babun, MATLAB (matrix laboratory), Continuous Integration Services (Wercker, Travis CI, Semaphore CI), docker}
           {\textsc{Familiar}}{AWS (Amazon Web Services), moltemplate, jekyll, middleman, grunt, gulp, Frameworks (Bourbon, Skeleton, neat) Markup Languages (Textile, HAML, Jade(pug)), Office-Suites (MS, OpenOffice, LibreOffice)}

\spacesubsection
\subsection{Operating Systems}
\cvdoubleitem{\textsc{Preferred}}{ArchLinux}
           {\textsc{Experienced}}{Windows (95, 2000, XP, 7, 8, 10), MacOS (10.7, 10.11, 10.12), Android (1.5, 1.6, 2.2.*, 2.3.*, 4.0.*, 4.4.*, 5.0.*, 6.0.*, 7.* ), Linux Distros (Ubuntu, Sabyon, Puppy, Manjaro, Debian, Red Hat (CentOS))}

\spacesubsection
\subsection{Opensource Contributions}
\spacesubsection

\cvdoubleitem{\textsc{Created}}{PixN ROM \& Kernel (AOSP based rom for the Xperia Z5) \\ HaoZeke's LineageOS }
           {\textsc{Mantained}}{Xperia Z5 LineageOS (14.*)}

\newpage

\subsection{Opensource Projects Created}
\spacesubsection

\cvdoubleitem{\textsc{zenYoda}}{Pandoc based, tup driven stand-alone multi format (revealJS, beamer etc.) presentation system with static site generation.}
           {\textsc{docuYoda}}{A document generation system based on pandoc and latexmk driven by gulp with yaml configuration and easy templating.}

\cvdoubleitem{\textsc{starDock}}{Docker compose based containerized self-updating setup for media hosting, with traefik for reverse proxying. Includes music, ebook and video acquisition and management. \\}
           {\textsc{pyQtNumSim}}{A Qt interface for verbose numerical methods assignments.}

\cvitem{\textsc{grimoire}}{Metalsmith and webpack based open source educational experiment with a strong focus on readability, equations and references.}


\spacesection
\section{Interests}
\subsection{Chemical Engineering}
\cvdoubleitem{\textsc{Experienced}}{Thermodynamics, Transport Phenomena, Mass Transfer, Heat Transfer, Molecular Dynamics (simulations)}
           {\textsc{Interested}}{Chemical Reaction Engineering (Statistical Interpretation), Process Control}

\spacesubsection
\subsection{Physics}
\cvdoubleitem{\textsc{Familiar}}{Statistical Thermodynamics, Density Functional Theory, Rare event sampling, Transition Path Theory, Markov State Models}
           {\textsc{Interested}}{Quantum Phenomena (Computing, Thermodynamics), Phase Transitions (Thermodynamics, Simulations), Chaos Theory, Spectroscopy, Entropy, Information Theory}

\spacesection
\section{Affiliations \& Accolades}
\subsection{Memberships}
\spacesubsection
\cventry{\textsc{2006--present}}{World Taekwondo}{Red Belt}{}{}{}
\cventry{\textsc{2009--present}}{XDA Developers}{Senior Member}{}{}{}
\cventry{\textsc{2015--present}}{IEEE (Institute of Electrical and Electronics Engineers)}{Student Member}{}{}{}
\cventry{\textsc{2015--present}}{APS (American Physical Society)}{Student Undergraduate Member}{}{}{}
\cventry{\textsc{2015--present}}{OSA (Optical Society of America)}{Student Member}{}{}{}
\cventry{\textsc{2015--present}}{IOP (Institute of Physics)}{Student Member}{}{}{}
\cventry{\textsc{2015--present}}{AICHE (American Institute Of Chemical Engineers)}{Student Member}{}{}{}
\cventry{\textsc{2017--present}}{IICHE (Indian Institute Of Chemical Engineers)}{Student Member}{}{}{}

\spacesubsection
\subsection{Awards}
\spacesubsection
\cventry{\textsc{December 2016}}{Photonics-2016}{Indian Institute Of Technology Kanpur}{Springer Best Student Paper Award}{Nonlinear-Optics Session}{}
\cventry{\textsc{2014--2015}}{IITG Zephyr Creative Writing}{Indian Institute Of Technology Guwahati}{First Prize}{}{}
\cventry{\textsc{2014--2015}}{Antaragni IITK-MUN GA-DISEC}{Indian Institute Of Technology Kanpur}{Best Speaker}{}{}

\spacesection

\newpage
\section{Publications}
{\color{color1}\textsc{Conference Proceedings}}

\nocite{*}
\printbibliography[heading=none, env=midbib, keyword=conference]


{\color{color1}\textsc{Preprints}}

\nocite{*}
\printbibliography[heading=none, env=midbib, keyword=preprint]

%{\color{color1}\textsc{Fictional Writings}}
%\printbibliography[heading=none, env=midbib, keyword=fiction]



%=============================
% this part is a simple cover letter
%\clearpage

%\recipient{Human Resources}{Some Company Ltd.\\Some Street 123\\12345 Some City} 
%\date{\today}
%\opening{Dear Sir or Madam,}
%\closing{Sincerely yours,}
%\enclosure[Attached]{curriculum vit\ae{}}

%\makelettertitle

%\lipsum[1-3] 

%\makeletterclosing


\end{document}
