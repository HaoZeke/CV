\documentclass[11pt,a4paper, final, factor=1100, stretch=18, shrink=18]{moderncv}

\usepackage{color}
\usepackage{fontspec}

% Stop bugging me about \es
\usepackage{xspace}
% moderncv themes
\moderncvstyle{classic}                             % style options are 'casual' (default), 'classic', 'banking', 'oldstyle' and 'fancy'
%\moderncvcolor{blue}                               % color options 'black', 'blue' (default), 'burgundy', 'green', 'grey', 'orange', 'purple' and 'red'

% the (custom) color which will be used in the cv
\definecolor{color1}{RGB}{1, 52, 64}

% scale the page layout (depreciated for more complicated stuff)
%\usepackage[scale=0.75]{geometry}

% change width of the column with the dates
\setlength{\hintscolumnwidth}{2.5cm}

%\PassOptionsToPackage[final, factor=1100, stretch=18, shrink=18 ]{microtype}
% The final option overrides global defaults. It greatly improves general appearance of the text. The stretch and shrink reduce bluriness[20,20 default]. The factor increases protrusion amount by 10%, [default 1000]
% Tracking allows for small caps, like in cites to be adjusted. The activate commands are to set protrusion.

\microtypecontext{spacing=nonfrench} 						% To preserve interword spacing via \nonfrenchspacing.

\SetExtraKerning[unit=space] 								% These produce more effects, from microtype with kerning.
{encoding={*}, family={bch}, series={*}, size={footnotesize,small,normalsize}}
{\textendash={400,400}, 								% en-dash, add more space around it
"28={ ,150}, 											% Left bracket, add space from right
"29={150, },											% Right bracket, add space from left
\textquotedblleft={ ,150}, 							% Left quotation mark, space from right
\textquotedblright={150, }} 							% Right quotation mark, space from left


\SetExtraKerning[unit=space]
{encoding={*}, family={qhv}, series={b}, size={large,Large}}
{1={-200,-200},
	\textendash={400,400}}

\SetTracking{encoding={*}, shape=sc}{40} 					% This is for better small caps with tracking and microtype.

\SetProtrusion{encoding={*},family={bch},series={*},size={6,7}}  % This enables better optical views.
{1={ ,750},2={ ,500},3={ ,500},4={ ,500},5={ ,500},
	6={ ,500},7={ ,600},8={ ,500},9={ ,500},0={ ,500}}

% Page settings
\usepackage{geometry}
\geometry{
	a4paper,
	%total={210mm,297mm},
	%l,r,t used to be 15mm
	left=15mm,
	right=15mm,
	top=15mm,
	bottom=10mm,
}

% required when changing page layout lengths
\AtBeginDocument{\recomputelengths}
\usepackage{xunicode}
\usepackage{xltxtra}
% \usepackage[utf8]{inputenc} (Xelatex expects UTF8 anyway)

% I like pretty logos
\usepackage{metalogo}

% Widen the letter spacing for logos
\setlogokern{La}{0.2pt}
\setlogokern{Xe}{0.2pt}
\setlogokern{eL}{0.2pt}
\setlogokern{Te}{0.2pt}
\setlogokern{aT}{0.2pt}
\setlogokern{eX}{0.2pt}

% Better Quotes (depreciated for this usage)
%\usepackage{epigraph}

% Poaching from Espresso's ug.tex

%%%%%%%%%%%%%%%%%%%%%%%%%%%%%%%%%%%%%%%%%%%%%%%%%%
%%%%%%%%%%%%%%%%%%%%%%%%%%%%%%%%%%%%%%%%%%%%%%%%%%
%%%%%%%%% New Commands and Environments %%%%%%%%%%
%%%%%%%%%%%%%%%%%%%%%%%%%%%%%%%%%%%%%%%%%%%%%%%%%%
%%%%%%%%%%%%%%%%%%%%%%%%%%%%%%%%%%%%%%%%%%%%%%%%%%
\newcommand{\es}{\mbox{\textsf{ESPResSo}}\xspace}

% german word break/hyphenation rules that's with ngerman (we use english)
\usepackage[british,UKenglish,USenglish,american]{babel}

% insert dummy text (used in the letter)
%\usepackage{lipsum}

% used for \begin{comment}...\end{comment}
\usepackage{verbatim}

% use guilllemets in bibliography (not german.) Also autostyle superceeded babel
\usepackage[autostyle=true]{csquotes}

\usepackage[
	sorting=none,
	minbibnames=8,
	maxbibnames=9,
	block=space
]{biblatex}

\bibliography{publications}


% get rid of the number-labels ([1], [2], etc.) in front of publications
\defbibenvironment{midbib}
{\list
	{}
	{
		\setlength{\leftmargin}{0mm}
		\setlength{\itemindent}{-\leftmargin}
		\setlength{\itemsep}{\bibitemsep}
		\setlength{\parsep}{\bibparsep}}
}
{\endlist}
{\item}


% add [DOI] and [PDF] fields at the end of each publication entry
\DeclareSourcemap{
	\maps[datatype=bibtex]{
		% the bibtex entry 'mydoi' gets mapped to 'usera'
		\map{
			\step[fieldsource=mydoi]
			\step[fieldset=usera, origfieldval]
		}
		% the bibtex entry 'mypdf' gets mapped to 'usera'
		\map{
			\step[fieldsource=mypdf]
			\step[fieldset=userb, origfieldval]
		}
	}
}

% [DOI] entries in publication
\DeclareFieldFormat{usera}{\color{color1}[\href{#1}{\textsc{doi}}]}
\AtEveryBibitem{
	% put [DOI] stuff at the end of a publication entry
	\csappto{blx@bbx@\thefield{entrytype}}{%
		\iffieldundef{usera}{
			% this gets invoked, once nothing is supplied
			% via the mypdf or mydoi value.
			% you could e.g. display a default thing here.
		}{\space\printfield{usera}}}
}

% [PDF] entries in publication
\DeclareFieldFormat{userb}{\color{color1}[\href{#1}{\textsc{pdf}}]}

\AtEveryBibitem{
	% put [DOI] stuff at the end of a publication entry
	\csappto{blx@bbx@\thefield{entrytype}}{\iffieldundef{userb}{}{\printfield{userb}}}
}

\renewcommand*{\mkbibnamegiven}[1]{%
	\ifitemannotation{highlight}
	{\textbf{#1}}
	{#1}}

\renewcommand*{\mkbibnamefamily}[1]{%
	\ifitemannotation{highlight}
	{\textbf{#1}}
	{#1}
}

% Minion Pro is used as the main font, if you don't
% have it installed uncomment this line or choose another pretty font.
% Literata is also included.
%\setmainfont[Numbers=OldStyle]{Minion Pro}

\setmainfont[
	Path = Fonts/MinionPro/,
	Numbers=OldStyle,
	Extension = .otf,
	UprightFont = *_Regular,
	ItalicFont = *_It,
	BoldFont = *_Bold,
	BoldItalicFont = *_BoldIt
]{MinionPro}

% Myriad Pro is used as the sans font, if you don't
% have it installed uncomment this line or choose another pretty font.
\setsansfont[
	Path = Fonts/MyriadPro/,
	Numbers=OldStyle,
	Extension = .otf,
	UprightFont = *_Regular,
	ItalicFont = *_It,
	BoldFont = *_Bold,
	BoldItalicFont = *_BoldIt,
	Ligatures=TeX,
	Scale=MatchLowercase]{MyriadPro}

% the moderncv package will populate a lot of the pdf meta-information.
% this can be used to change some of these infos.
\AfterPreamble{\hypersetup{
		pdfcreator={XeLaTeX},
		pdftitle={Rohit Goswami's CV}
	}}

% for the icons (telephone, globe). I found the icons provided by the
% fontawesome package prettier than the standard moderncv icons.
\defaultfontfeatures{
	Path = Fonts/FontAwesome/ }
\usepackage{fontawesome}

% personal data
\firstname{Rohit}
\familyname{Goswami}
\quote{``An unproblematic state is a state without creative thought. It's other name is Death.''\\-- David Deutsch}

% \faEnvelope \faPhone \faGithub \faGlobe

% \address{House No. 646, 35th Lane}{IIT Kanpur, 208016}

% I use the extrainfo command for additional information, since I
% want to use custom icons and have finer control over spacing.
\extrainfo{
% \faPhone\hspace{0.3em}\href{tel:919935135006}{\ttfamily +91 9935135006}\\
{\small\faEnvelope}\hspace{0.3em}\href{mailto:rgoswami@ieee.org}{\ttfamily rgoswami@ieee.org}\\
{\small\faGithub}\hspace{0.3em}\href{https://github.com/haozeke}{\ttfamily haozeke}\\
{\small\faTwitter}\hspace{0.3em}\href{https://twitter.com/rg0swami}{\ttfamily rg0swami}\\
\faGlobe\hspace{0.3em}\href{https://rgoswami.me}{\ttfamily rgoswami.me}
}


% picture, resized to a height of 84pt
\photo[84pt]{Picture/rohit}

% spacing before (sub)sections
\newcommand{\spacesection}{\vspace{0.4cm}}
\newcommand{\spacesubsection}{\vspace{0.2cm}}


%===========================
\begin{document}

% Adapted from https://tex.stackexchange.com/a/170219/130845
\begin{tikzpicture}[remember picture,overlay]
      \node[anchor=north, yshift=-0.25cm] at (current page.north) {\underline{Last
		  updated on \today}};
\end{tikzpicture}

\maketitle

\section{Personal Data}
\cvitem{Name}{Rohit Goswami}
\cvitem{Date Of Birth}{10.08.1995}
\cvitem{Birthplace}{Brookhaven, New York, United States of America}

\spacesection
\section{Work Experience}

\cventry{\textsc{2019--present}}{Faculty of Physical Sciences}{University of Iceland}{}{Graduate Researcher}{I am working with Prof. Hannes Jónsson on Bayesian analysis and machine learning for \emph{ab-inito} quantum chemistry.}
\cventry{\textsc{2019--present}}{Department of Chemistry}{Indian Institute Of Technology, Kanpur}{}{Senior Project Associate}{I am affiliated to the Femtolab under the project ``Femtosecond Laser Approaches to Quantum Information and Quantum Computation''}
\cventry{\textsc{2018--2019}}{Department of Chemical Engineering}{Indian Institute Of Technology, Kanpur}{}{Project Associate}{I was associated with the Computational Nanoscience group. Over the course of two centrally funded projects, ``Nucleation On Nanostructured Surfaces Computer Simulation      Studies (SPO/DST/CHE/2017294)'' and ``Advanced Computation Research and Education (SPO/MHRD/CC/20130176)'':
	\begin{itemize}
		\item I worked on the
		      implementation of an enhanced version of the CHILL (CHILL+) algorithm for tracking
		      ice types.
		\item Designed a linear discriminant analysis technique for near-surface ice structure determination which is undergoing rigorous testing
		\item Implemented a graph based network connectivity model for ice structures
		\item Spearheaded the development of High Performance GPU accelerated molecular dynamics simulation analysis tools
		\item Worked on the determination of optimal GPU cluster configurations
		\item Designed and administered academic outreach websites
	\end{itemize}  }

\spacesection
\section{Education}

%\cventry{\textsc{1948}}{Ph.D. Chemistry}{University Extension}{New York}{\emph{United States}}{}
%\cventry{\textsc{1939--1941}}{M.Sc. Chemistry}{University Extension}{New York}{\emph{United States}}{}
\cventry{\textsc{2019--present}}{Graduate studies}{University of Iceland}{Reykjavík}{\emph{Iceland}}{Current GPA: $9.0$}
\cventry{\textsc{2014--2018}}{B.Tech. Chemical Engineering}{Harcourt Butler Technical University}{Kanpur}{\emph{India}}{First Division (\textsc{Project: } Gas Sweetening Plant Design)}
\cventry{\textsc{2011--2013}}{Intermediate (AISSCE)}{Delhi Public School Kalyanpur}{Kanpur}{\emph{India}}{$87.2\%$ Central Board of Secondary Education (CBSE)}
\cventry{\textsc{2009--2011}}{High School (AISSE)}{Delhi Public School Kalyanpur}{Kanpur}{\emph{India}}{$9.8$ Cumulative Grade Point Average (CGPA) in Central Board of Secondary Education (CBSE)}

\spacesection
\section{Voluntary Positions}

\cventry{\textsc{June-July 2020}}{Online Data Carpentry Workshop}{SADiLaR, South Africa}{}{Lead Instructor}{Taught the basics of R, OpenRefine, and some data wrangling to graduate students in the social sciences over three days.}
\cventry{\textsc{June-July 2020}}{Data Carpentry Ecology Workshop}{Biotech Partners}{}{Leading Instructor}{Taught the basics of Python and assisted with shell lessons for high school students over three days, with a follow up mentoring program.}
\cventry{\textsc{2020-2021}}{Executive Student Council}{American Institute of Chemical Engineers}{}{Publications Webmaster}{Am tasked with managing the publications committee web resources.}
\cventry{\textsc{May 2020}}{Helper}{CodeRefinery Mega Workshop}{}{}{}
\cventry{\textsc{April-May 2020}}{CS106A - Code in Place}{Stanford University}{}{Section Leader (TA)}{As
  part of the special COVID-19
  \href{https://compedu.stanford.edu/codeinplace/announcement/}{\ttfamily code-in-place
	initiative}, I worked as a section leader (teaching assistant). The course
  covered the fundamentals of computer programming using Python and was built off the first half of CS106A. Also moderated and participated in an AMA session on ``Machine Learning for the Physical Sciences'' and taught a workshop for the other section leaders entitled ``Functional Python Packaging with Nix''}

\cventry{\textsc{2019--present}}{IEEE P1940}{IEEE Standards Committee}{}{Working group member}{Am actively engaged in working with stake holders in industry and academia to create a collection of standard profiles that define integration of authentication services with ISO 8583 used for financial transactions.}

\cventry{\textsc{2019--present}}{R Novice Inflammation}{The Carpentries}{}{Maintainer}{As a maintainer for the Software Carpentries lesson on R, I work with the community to make sure that lessons stay up-to-date, accurate, functional and cohesive. }

\cventry{\textsc{2019--present}}{CarpentryCon 2020}{The Carpentries}{}{Program Committee co-chair \& Website subcommittee member}{Working for an international conference with diverse leads from across the world, as part of the program committee I reached out to keynote speakers and managed the overall schedule. Wrote content with the website subcommittee and also contributed due to my web development expertise.}

\cventry{\textsc{2019--present}}{Univ.ai}{Earth2Orbit Analytix Private Limited}{}{Teaching Fellow and Developer}{Tested course-content and developed interactive labs to work with an online cohort of students. Am presently teaching labs and mentoring small batches. I also work with the front and backend teams to facilitate workflows including shopify stores and NodeJS authentication systems.}

\cventry{\textsc{2018--2019}}{Animal Welfare Group}{Indian Institute of Technology Kanpur}{}{Member and Web-developer}{Have worked with student bodies to rescue and care for local animals. Also designed and maintain a site with ReactJS to enhance knowledge dissemination.}

\spacesection
\section{Undergraduate Experience}

\spacesubsection
\subsection{Internships}

\cventry{\textsc{2017--2018}}{Dr. Debojit Chakrabarty}{Keva Fragrances Ltd, Mumbai}{}{R\&D Industrial Intern}{Modeling complex multi-component perfumes in a predictive method via experimental and theoretical considerations. In collaboration with Prof. Rajdip Bandyopadhyaya of the ChemE Dept. at IIT Bombay.
}
\cventry{\textsc{Summer 2017}}{Prof. Sibasish Ghosh}{The Institute of Mathematical Sciences, Chennai}{}{Visiting Scholar}{Discussed computational techniques for the simulation and understanding of quantum tomography.}
\cventry{\textsc{Summer 2017}}{Prof. Nisanth Nair}{Indian Institute Of Technology Kanpur}{}{SURGE Scholar}{An exploratory project to understand and deal with bottlenecks in computational chemistry, the major objectives were to investigate hybridization of existing code via OpenMP and MPI.
	\\~\\
	\textsc{Poster: }Development of Computational Tools for Free Energy Calculations of Chemical Reactions}
\cventry{\textsc{Summer 2016}}{Dr. Rajarshi Chakrabarti}{Indian Institute Of Technology Bombay}{}{Research Intern}{ Retooled a server with ArchLinux and also simulated patchy colloids (Janus Particles). \\~\\
	\textsc{Project Report: }Computational Survey of Coarse Grained Soft Matter Molecular Dynamics Simulations
}

\spacesubsection
\subsection{Volunteer Work}

\cventry{\textsc{2017--2018}}{ChemE Herald}{Harcourt Butler Technical University, Kanpur}{}{Editor-in-Chief}{Inaugurated and managed an interdisciplinary technical newsletter.}

\cventry{\textsc{2017--2018}}{HBTU-MUN 2018}{}{}{Secretary General}{Designed a ReactJS based static website, with Trello backed user registration, also performed outreach pre-events to raise awareness and participation, in addition to overseeing the working of the executive board.}

\cventry{\textsc{2016--2017}}{HBTU-MUN 2017}{}{}{Executive Board Chairperson}{Designed a Jekyll based static website and ensured adherence to standard MUN rules as Chairperson.}

% Sadly this was hit by anti-aircraft guns
% \cventry{\textsc{2016--2017}}{Interface 2017}{Harcourt Butler Technical University, Kanpur}{}{Technical Cell Head}{Organizing and inspiring students to work towards the success of the Department's techno-cultural fest and seminars.}

\cventry{\textsc{2014--2016}}{The Curiosity Magazine}{Harcourt Butler Technical University, Kanpur}{}{Editor-in-Chief}{Managed a diverse team of student content writers and also later typeset a spin-off multi-lingual newsletter in \XeLaTeX.}

\spacesection
% Consider downgrading to the bottom
\section{Technical Skills}
\subsection{Programming Languages}

\cvdoubleitem{\textsc{Experienced}}{C++, Python, R, FORTRAN, Shell (zsh, bash),  OpenMP, OpenMPI, Tcl, CSS, JS, HTML, Sass, C}
{\textsc{Familiar}}{Ruby, Julia, Java, Haskell, Matlab, Golang, ReactJS, Node, CUDA}

\spacesubsection
\subsection{Projects}

\cvdoubleitem{\textsc{Experienced}}{d-SEAMS, Android (Cyanogen, LineageOS, AOSP), Web-Design (static), ArchLinux}
{\textsc{Familiar}}{Linux Kernel (Android)}

\spacesubsection
\subsection{Simulation Projects}

\cvdoubleitem{\textsc{Experienced}}{\es \space(Extensible Simulation Package for Research on Soft matter), LAMMPS (Large-scale Atomic/Molecular Massively Parallel Simulator), OVITO, AiiDA}
{\textsc{Familiar}}{OpenFOAM, GROMACS (GROningen MAchine for Chemical Simulations), VMD (Visual Molecular Dynamics), CPMD (Car-Parrinello Molecular Dynamics)}

\spacesubsection
\subsection{Tools}

\cvdoubleitem{\textsc{Experienced}}{\XeLaTeX, pandoc, Git (version control), tmux, ssh, Vim, Sublime Text Editor 3, gnuplot, gadfly, bspwm (tiling window manager), mosh, babun, MATLAB (matrix laboratory), Continuous Integration Services (Wercker, Travis CI, Semaphore CI), docker}
{\textsc{Familiar}}{AWS (Amazon Web Services), moltemplate, jekyll, middleman, grunt, gulp, Frameworks (Bourbon, Skeleton, neat) Markup Languages (Textile, HAML, Jade(pug)), Office-Suites (MS, OpenOffice, LibreOffice)}

% \spacesubsection
% \subsection{Operating Systems}
% \cvdoubleitem{\textsc{Preferred}}{ArchLinux}
% {\textsc{Experienced}}{Windows (95, 2000, XP, 7, 8, 10), MacOS (10.7, 10.11, 10.12), Android (1.5, 1.6, 2.2.*, 2.3.*, 4.0.*, 4.4.*, 5.0.*, 6.0.*, 7.* ), Linux Distros (Ubuntu, Sabyon, Puppy, Manjaro, Debian, Red Hat)}

\spacesubsection
\subsection{Opensource Contributions}

\cvdoubleitem{\textsc{Created}}{PixN ROM \& Kernel (AOSP based rom for the Xperia Z5) \\ HaoZeke's LineageOS }
{\textsc{Mantained}}{Xperia Z5 LineageOS (14.*)}

\subsection{Opensource Projects Created}

\cvitem{\textsc{d-SEAMS}}{An open-source, community supported engine for the analysis of molecular dynamics trajectories (co-creator)}

\cvdoubleitem{\textsc{zenYoda}}{Pandoc based, tup driven stand-alone multi format (revealJS, beamer etc.) presentation system with static site generation.}
{\textsc{docuYoda}}{A document generation system based on pandoc and latexmk driven by gulp with yaml configuration and easy templating.}

\cvdoubleitem{\textsc{starDock}}{Docker compose based containerized self-updating setup for media hosting, with traefik for reverse proxying. Includes music, ebook and video acquisition and management. \\}
{\textsc{pyQtNumSim}}{A Qt interface for verbose numerical methods assignments.}

\cvdoubleitem{\textsc{grimoire}}{Metalsmith and webpack based open source educational experiment with a strong focus on readability, equations and references. \\}
{\textsc{rgoswami.me}}{A hugo-blog template meant to be used with Emacs orgmode}

% \spacesection
% \section{Interests}
% \subsection{Chemical Engineering}
% \cvdoubleitem{\textsc{Experienced}}{Thermodynamics, Transport Phenomena, Mass Transfer, Heat Transfer, Molecular Dynamics (simulations)}
%            {\textsc{Interested}}{Chemical Reaction Engineering (Statistical Interpretation), Process Control}

% \spacesubsection
% \subsection{Physics}
% \cvdoubleitem{\textsc{Familiar}}{Statistical Thermodynamics, Density Functional Theory, Rare event sampling, Transition Path Theory, Markov State Models}
%            {\textsc{Interested}}{Quantum Phenomena (Computing, Thermodynamics), Phase Transitions (Thermodynamics, Simulations), Chaos Theory, Spectroscopy, Entropy, Information Theory}

\spacesection
\section{Affiliations \& Accolades}

\spacesubsection
\subsection{Memberships}

\cventry{\textsc{2014--present}}{OSA (Optical Society of America)}{}{Student Member $\to$ Early Career Member (2018)}{}{}
\cventry{\textsc{2015--present}}{AIChE (American Institute Of Chemical Engineers)}{}{Student Member $\to$ Young Professional (2018)}{}{}
\cventry{\textsc{2015--present}}{APS (American Physical Society)}{}{Student Undergraduate Member $\to$ Early Career Member (2019)}{}{}
\cventry{\textsc{2015--present}}{IEEE (Institute of Electrical and Electronics Engineers)}{}{Student Member $\to$ Early Career Member (2018)}{}{}
\cventry{\textsc{2015--present}}{IOP (Institute of Physics)}{}{Student Member (2018) $\to$ Member (2019)}{}{}
\cventry{\textsc{2006--present}}{World Taekwondo}{}{Red Belt}{}{}
\cventry{\textsc{2009--present}}{XDA Developers}{}{Senior Member}{}{}
% \cventry{\textsc{2019--present}}{SIGHPC (Special Interest Group for High Performance Computing) ACM Chapter}{Professional Member}{}{}{}
% \cventry{\textsc{2019--present}}{SIGHPC-Education ACM Chapter}{Professional Member}{}{}{}
\cventry{\textsc{2019--present}}{AAAI (Association for the Advancement of Artificial Intelligence)}{}{Professional Member}{}{}
\cventry{\textsc{2019--present}}{ACM (Association for Computing Machinery)}{}{Professional Member}{}{Also part of the SIGHPC (Special Interest Group for High Performance Computing) \& SIGHPC-Education}
\cventry{\textsc{2019--present}}{ASAPBio (Accelerating Science and Publication in biology)}{}{Ambassador}{}{}
\cventry{\textsc{2019--present}}{IChemE (Institute of Chemical Engineers)}{}{Associate Member}{}{}
\cventry{\textsc{2019--present}}{IIChE (Indian Institute of Chemical Engineers)}{}{Life Associate Member}{}{}
\cventry{\textsc{2019--present}}{IEEE IAS (Industrial Applications Society)}{}{Member}{}{}
\cventry{\textsc{2019--present}}{IEI (The Institution of Engineers [India])}{}{Associate Member}{}{}
\cventry{\textsc{2019--present}}{InRaSS (Indian Radio Science Society)}{}{Student Member}{}{}
\cventry{\textsc{2019--present}}{OSI (Open Source Initiative)}{}{Individual Member}{}{}
\cventry{\textsc{2019--present}}{OSI (Optical Society of India)}{}{Life Fellow}{}{}
\cventry{\textsc{2019--present}}{SPIE (Society of Photo-Optical Instrumentation Engineers)}{}{Early Career Professional}{}{}
\cventry{\textsc{2019--present}}{The Carpentries}{}{Certified Instructor}{}{}
\cventry{\textsc{2019--present}}{URSI (Union Radio-Scientifique Internationale)}{}{Corresponding Member}{}{}
\cventry{\textsc{2020--present}}{Heterodox Academy}{}{Graduate Affiliate Member}{}{}
\cventry{\textsc{2020--present}}{IGDORE (Institute for Globally Distributed Open Research and Education)}{}{Researcher-in-training}{}{}
\cventry{\textsc{2020--present}}{COS (Center for Open Sciences)}{}{Ambassador}{}{}
\cventry{\textsc{2020--present}}{ims (Institute of Mathematical Statistics)}{}{Student Member}{}{}
\cventry{\textsc{2020--present}}{Bernoulli Society}{}{Student Member}{}{}
\cventry{\textsc{2020--present}}{RSS (Royal Statistical Society)}{}{E-Student Member}{}{}
\cventry{\textsc{2020--present}}{Computing at School}{}{Member}{}{}

\spacesubsection
\subsection{Awards}

\cventry{\textsc{December 2016}}{Photonics-2016}{Indian Institute Of Technology Kanpur}{Springer Best Student Paper Award}{Nonlinear-Optics Session}{}
\cventry{\textsc{2014--2015}}{IITG Zephyr Creative Writing}{Indian Institute Of Technology Guwahati}{First Prize}{}{}
\cventry{\textsc{2014--2015}}{Antaragni IITK-MUN GA-DISEC}{Indian Institute Of Technology Kanpur}{Best Speaker}{}{}

\subsection{Reviews}

\cventry{\textsc{2018--present}}{Journal Of Open Source Software}{Reviewer}{}{}{I review submissions pertaining to molecular dynamics,virtualization, HPC, web platforms, finite element methods, optimization and computer geometry written primarily in C++, Rust, FORTRAN, Julia, Javascript. I am also listed for Python and R submissions. Specifically:
	\begin{itemize}
		\item \href{https://joss.theoj.org/papers/76800aa00f4d57b1695c876c0b936ba3}{DEPP - Differential Evolution Parallel Program}
		\item \href{https://joss.theoj.org/papers/553250d815e1990db1b89c742854c71a}{RHEOS - A Julia package for Rheology Data Analysis}
		\item \href{https://joss.theoj.org/papers/25d51faf73cc17ae3affb51b787bbe18}{Computing diffusion coefficients in macromolecular simulations: the Diffusion Coefficient Tool for VMD}
		\item \href{http://joss.theoj.org/papers/093a37e7d698d3350b2d4e564dcfb69b}{Simple-Web-Server: a fast and flexible HTTP/1.1 C++ client and server library}
		\item \href{http://joss.theoj.org/papers/033b265ce2f8562dc1148b534a09275c}{HyperNaut: a navigator for the hyperbolic plane}
		\item \href{http://joss.theoj.org/papers/fc7f571e455ef955d1e5dd271343723f}{The Biddy BDD package}
		\item \href{http://joss.theoj.org/papers/f3b70136a43eb28b9327ec1ce42533c2}{Prest: Open-Source Software for Computational Revealed Preference Analysis}
	\end{itemize}
}
\cventry{\textsc{2019--present}}{PeerJ - Life \& Environment}{Reviewer}{}{}{}
\cventry{\textsc{2019--present}}{PeerJ - Computer Science}{Reviewer}{}{}{}
\cventry{\textsc{2019--present}}{PeerJ - Organic Chemistry}{Reviewer}{}{}{}
\cventry{\textsc{2019--present}}{PLOS ONE}{Reviewer}{}{}{}
\cventry{\textsc{2020--present}}{APS Physical Review Letters}{Reviewer}{}{}{}
\cventry{\textsc{2020--present}}{SPIE Journal of Micro/Nanolithography, MEMS, and MOEMS}{Reviewer}{}{}{}
% {In keeping with the PeerJ commitment to taking authorship of comments, I have reviewed:
% 	\begin{itemize}
% 		\item \href{https://peerj.com/submissions/40173/reviews/557309/review/}{Gini coefficients for measuring the distribution of sexually transmitted infections among individuals with different levels of sexual activity}
% 	\end{itemize}
% }

\spacesection
\section{Publications}
 {\color{color1}\textsc{Journals}}

\nocite{*}
\printbibliography[heading=none, env=midbib, keyword=journal]

{\color{color1}\textsc{Conference Proceedings}}

\nocite{*}
\printbibliography[heading=none, env=midbib, keyword=conference]


{\color{color1}\textsc{Preprints}}

\nocite{*}
\printbibliography[heading=none, env=midbib, keyword=preprint]

\spacesection
\section{Conference Records}

\subsection{Posters}
\cventry{\textsc{March 2020}}{Ultrafast Insights for Predictive Fragrance Compounding}{ACS Spring 2020 National Meeting}{R. Goswami, A. K. Rawat, D. Chakrabarty, and D. Goswami}{Accepted}{}
\cventry{\textsc{December 2019}}{Qubit Network Barriers to Deep Learning}{IEEE WRAP-2019}{R. Goswami, A. Goswami, and D. Goswami}{}{}
\cventry{\textsc{March 2019}}{Space Filling Curves: Heuristics For Semi Classical Lasing Computations}{URSI Asia-Pacific Radio Science Conference (AP-RASC 2019)}{R. Goswami, A. Goswami, and D. Goswami}{}{}
\cventry{\textsc{December 2018}}{FDTD Numerical Computations for Ultrafast Non-linear Optics}{Photonics-2018}{R. Goswami, A. Goswami, and D. Goswami}{}{}

\subsection{Oral Presentations}
\cventry{\textsc{July 2020}}{LFortran: Interactive LLVM-based Fortran Compiler for Modern Architectures}{FortranCon 2020}{O. Čertík, N. Maan, A. Pandey, M. Curcic, P. Brady, Z. Jibben, N. Carlson, R. Goswami, A. Shahmoradi and A. Markus}{Presented by Ondřej}{}
\cventry{\textsc{December 2019}}{Process Safety in terms of Latent Dirichlet Allocations}{72nd Annual Session of of the Indian Institute of Chemical Engineers, CHEMCON-2019}{R. Goswami}{Accepted}{}
\cventry{\textsc{October 2019}}{Semi-Supervised Approaches to Ultrafast Optimal Control Theory}{43rd Symposium of the Optical Society of India, International Conference on Optics \& Electro-Optics}{R. Goswami, A. Goswami and D. Goswami}{}{}
\cventry{\textsc{December 2016}}{Quantum Distributed Computing with Shaped Laser Pulses}{13th International Conference on Fiber Optics and Photonics}{R. Goswami and D. Goswami}{}{}

\spacesection
\section{Workshops}

\cventry{\textsc{July 2020}}{AiiDA Virtual Tutorial}{EPFL, Lausanne, Switzerland}{A four day online in-depth tutorial by core and plugin developers covering the nuances of reproducible workflows for AiiDA with example Quantum ESPRESSO calculations on Quantum Mobile virtual machines.}{}{}
\cventry{\textsc{June 2020}}{Mathematical Methods of Modern Statistics 2}{CIRM Virtual Event}{A five day online event from the Centre International de Rencontres Mathématiques on recent statistical advances relating to the analysis of high dimensional data from frequentist and Bayesian perspectives.}{}{}
\cventry{\textsc{May 2020}}{ALCF Computational Performance Workshop}{Online}{A three day workshop on the hardware and software improvements organized by Argonne Leadership Computing Facility.}{}{}
\cventry{\textsc{October 2019}}{TriangleSCI 2019}{Invited to the Triangle Scholarly
	Communication Institute}{A week long fully-funded incubator to discuss
	actionable goals towards Bringing Equity and Diversity to Peer Review. This
	was undertaken as part of the larger discussion on Equity in Scholarly
	Communications}{\textbf{declined to attend}}{}
\cventry{\textsc{May-June 2019}}{Artificial Intelligence}{E \& ICT Academy, IIT Kanpur}{A four week course on AI foundations culminating in a time-series prediction project}{}{}
\cventry{\textsc{June 2019}}{AI Foundations Certificate Course}{univ.ai}{A summer school taught in-person by faculty from Harvard and UCLA, culminating in a computer vision and neural network based identification project}{}{}
\cventry{\textsc{July 2019}}{Rare Events Summer School}{Indian Institute of Science, Bangalore}{A short course consisting of lectures and hands-on sessions by experts in the field, organized by Prof. Baron Peters}{}{}
\subsection{Short Courses}
\cventry{\textsc{September 23\textsuperscript{rd} 2019}}{Surface Area and Porous Material Characterization}{Dept. of ChemE, IIT Kanpur}{An intensive day long course on the basics of experimental classification and DFT methods for pore distribution by Dr. Martin Thomas from Anton-Paar}{}{}
\cventry{\textsc{September 21\textsuperscript{st} 2019}}{OpenACC GPU Bootcamp}{Chemistry Department, IIT Kanpur}{Day long programming session and discussion covering the acceleration of Institute in-house code facilitated by a Senior Nvidia Solution Architect (Mr. Bharatkumar) and Prof. Debabrata Goswami}{}{}

\spacesection
\section{Certifications}

\subsection{Coursera}
\cventry{\textsc{March 2020}}{Deep Learning
  Specialization}{deeplearning.ai}{}{$99.5\%$}{
  \href{https://www.coursera.org/account/accomplishments/specialization/Q3DLMMZJ42TR}{The specialization can be verified by ID}
  \href{https://www.coursera.org/account/accomplishments/specialization/Q3DLMMZJ42TR}{\ttfamily
	Q3DLMMZJ42TR}. This signifies the completion of the following five courses:
	\begin{itemize}
		\item
		\href{https://www.coursera.org/account/accomplishments/verify/WFMW5VPBCKVJ}{Neural
		Networks and Deep Learning} - $\mathbf{100.0\%}$
		\item
		\href{https://www.coursera.org/account/accomplishments/records/9NJJAPSMK59A}{Improving
		Deep Neural Networks: Hyperparameter tuning, Regularization and
		Optimization} - $\mathbf{99.2\%}$
		\item
		\href{https://www.coursera.org/account/accomplishments/records/QWTW54L3CEDC}{Structuring
		Machine Learning Projects} - $\mathbf{98.3\%}$
		\item
		\href{https://www.coursera.org/account/accomplishments/verify/A23X8Q7QHJPV}{Convolutional
		Neural Networks} - $\mathbf{100.0\%}$
		\item
		\href{https://www.coursera.org/account/accomplishments/records/D8JCTU57D6CN}{Sequence
		Models} - $\mathbf{100.0\%}$
	\end{itemize}
}

% Consider moving to be below Education
\spacesubsection
\subsection{NPTEL Courses}
\cventry{\textsc{Jan-Apr 2019}}{Graph Theory}{IISER Pune}{}{$55\%$}{\href{https://nptel.ac.in/noc/E_Certificate/noc19-ma13/NPTEL19MA13S21460067191106132.jpg}{License: NPTEL19MA13S21460067}}
\cventry{\textsc{Aug-Sep 2018}}{Computational Chemistry and Classical Molecular Dynamics}{IIT Bombay}{Elite}{$77\%$}{\href{https://nptel.ac.in/noc/social_cert/noc18-cy13/NPTEL18CY13S214401271810100922.jpg}{License: NPTEL18CS13S21440127}}
\cventry{\textsc{Aug-Sep 2018}}{Introduction to Parallel Programming in OpenMP}{IIT Madras}{}{$40\%$}{\href{https://nptel.ac.in/noc/social_cert/noc18-cs55/NPTEL18CS55S114401221810100922.jpg}{License: NPTEL18CS55S11440122}}
\cventry{\textsc{Jan-Apr 2018}}{Quantum Computing}{IIT Kanpur}{Elite}{$65\%$}{\href{https://nptel.ac.in/noc/social_cert/noc18-cy07/NPTEL18CY07S44800241810011238.jpg}{License: NPTEL18CY07S4480024}}

%{\color{color1}\textsc{Fictional Writings}}
%\printbibliography[heading=none, env=midbib, keyword=fiction]

%=============================
% this part is a simple cover letter
%\clearpage

%\recipient{Human Resources}{Some Company Ltd.\\Some Street 123\\12345 Some City}
%\date{\today}
%\opening{Dear Sir or Madam,}
%\closing{Sincerely yours,}
%\enclosure[Attached]{curriculum vit\ae{}}

%\makelettertitle

%\lipsum[1-3]

%\makeletterclosing


\end{document}

%%% Local Variables:
%%% mode: latex
%%% TeX-master: t
%%% End:
